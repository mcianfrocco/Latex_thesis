\chapter{Biochemical and structural support for a general model of promoter binding by TFIID}

The structural studies presented in Chapters 2 \& 3 have provided the foundation for understanding the nature of TFIID's interaction with promoter DNA. However, in order to arrive at a more complete description of TFIID bound to promoter DNA, biochemical and structural studies were pursued in an attempt to propose a model of TFIID-TFIIA-SCP complex formation. To facilitate the biochemical studies, we engaged in a wonderful collaboration with James T. Kadonaga's laboratory at University of California - San Diego, where George A. Kassavetis performed all footprinting experiments shown in this chapter. The synergy between structural studies of TFIID-TFIIA-SCP and biochemical footprinting have helped construct the first model of human TFIID's regulated-interaction with promoter DNA.\\
\indent This chapter will address potential mechanisms of promoter binding by TFIID through a variety of experimental designs. First, detailed footprinting analysis using DNase I and methidium-propyl-EDTA-Fe(II) reveal the extent of contacts that TFIID makes with promoter DNA in a TFIIA-dependent and -independent fashion. These footprinting results will be extended by testing the structural effect that core promoter mutations have on TFIID's structure. After developing a detailed model of TFIID-TFIIA-DNA, the canonical conformation will be probed to investiage the DNA binding activity, if any, that this conformation exhibits. Finally, cryo-EM structural analysis and footprinting will establish that the rearranged conformation is the TFIIB-binding conformation of TFIID, suggesting that the rearranged state is the conformation that loads RNAPII at the TSS. These results have been synthesized to propose a model that addresses the interplay between TFIID's conformational landscape and DNA binding (Figure~\ref{fig:Fig4.13})
  
\section{Detailed footprinting of TFIID-TFIIA-SCP}
\begin{figure}
\centering
\includegraphics[width=0.9\textwidth]{../Ch4_figs/Fig4.1.eps}
\caption[TFIID exhibits TFIIA-independent and TFIIA–dependent interactions with SCP DNA]{TFIID exhibits TFIIA-independent and TFIIA–dependent interactions with SCP DNA. DNase I (A) and MPE-Fe (B) footprinting of TFIID-SCP and TFIID-IIA-SCP. (C) B-DNA model of the SCP with positions of core promoter elements (top), DNase I, and MPE-Fe protection patterns (-66 to +55 shown) color-coded for TFIID-SCP and TFIID-IIA-SCP (four bottom rows).  White basepairs indicate protection (also marked by black lines), blue surfaces indicate partial digestion and red surfaces indicate complete digestion by DNase I, and pink indicates digestion by MPE-Fe.}
\label{fig:Fig4.1}
\end{figure}
The cryo-EM data of the rearranged conformation of the TFIID-TFIIA-SCP complex suggest that the downstream DNA core promoter elements MTE and DPE are bound by lobe C, whereas the TATA box and possibly the Inr are bound by lobe A. To test and extend this model, we performed footprinting analyses of the TFIID-SCP complex in the presence or absence of TFIIA.\\
\indent DNase I footprinting of TFIID-SCP showed an extended region of protection from -7 to +41, which is in agreement with previous footprinting data on TFIID-SCP (Figure~\ref{fig:Fig4.1}A) \cite{Juven-Gershon_1249}. The addition of TFIIA did not alter the downstream interactions of TFIID with the Inr, MTE, and DPE core promoter elements, but did, however, result in a substantial increase in the binding of TFIID to the TATA box and flanking DNA sequences from -38 to -20 (Figures~\ref{fig:Fig4.1}A \& \ref{fig:Fig4.2}A \& B). DNase I footprinting analysis of both DNA strands in the absence or presence of TFIIA revealed distinct patterns of protection and cleavage throughout the core promoter region. The results indicate that the DNA sequence between the Inr and MTE/DPE sites exhibits a phasing in DNase I sensitivity in which one face of the SCP DNA between Inr and MTE-DPE is susceptible to DNase I cleavage, whereas the opposite face remains protected (Figure~\ref{fig:Fig4.1}C). Furthermore, the accessible face of the DNA exhibits DNase I hypersensitivity upon binding of TFIID (Figure~\ref{fig:Fig4.1}A). \\
\begin{figure}
\centering
\includegraphics[width=1\textwidth]{../Ch4_figs/Fig4.2.eps}
\caption[Line profiles for DNase I footprinting gels of TFIID-SCP and TFIID-TFIIA-SCP]{Line profiles for DNase I footprinting gels of TFIID-SCP and TFIID-TFIIA-SCP. (A) Line profile traces from DNase I footprinting gels for 5’-label upstream (A) and downstream (B). Bold basepair numbers indicate hypersensitive sites. Line profile traces from MPE-Fe footprinting gels for 5’-label upstream (C) and 5’-label downstream (D).}
\label{fig:Fig4.2}
\end{figure}
\indent To obtain high resolution data on the interactions of TFIID with the core promoter, we carried out footprinting analyses with methidiumpropyl-EDTA-Fe(II) (MPE-Fe), an intercalating agent that delivers Fe(II) for oxidation of the deoxyribose phosphate backbone of DNA and provides single bp resolution of protein-DNA contacts \cite{Hertzberg_3897,Papavassiliou_3156,Va_3928}. The MPE-Fe footprinting data on TFIID-SCP and TFIID-TFIIA-SCP revealed the extensive and continuous interaction of TFIID with DNA from the DPE through the Inr as well as the TFIIA-dependent protection of 8 bp (-31 to -24) of DNA encompassing the TATA box (Figures~\ref{fig:Fig4.1}B \& \ref{fig:Fig4.2}C \& D). The strong stimulation of TFIID binding to the TATA box is consistent with the results of previous studies on the binding of TBP to DNA \cite{Geiger_2949,Kim_3377,Kim_3416,Nikolov_3177}. The region of DNase I protection observed on only one face of the helix between the MTE/DPE and Inr also shows continuous protection from MPE-Fe(II) cleavage on both DNA strands, which is likely due to the inhibition of the DNA unwinding, necessary for MPE intercalation, that would occur as a result of protein bound to one side of the helix \cite{Uchida_3659}. Thus, the DNase I and MPE-Fe footprinting results, summarized in Figure~\ref{fig:Fig4.1}C, provide insight into the TFIID-DNA contacts that complements the cryo-EM data and contributes to the placement of the core promoter DNA on the TFIID structure. \\
\begin{figure}
\centering
\includegraphics[width=1\textwidth]{../Ch4_figs/Fig4.4.eps}
\caption[Structural model of DNase I and MPE-Fe footprinting results within the ternary TFIID-TFIIA-SCP(-66) complex]{Structural model of DNase I and MPE-Fe footprinting results within the ternary TFIID-TFIIA-SCP(-66) complex. Promoter DNA for SCP(-66) docked into the TFIID-TFIIA-SCP(-66) map (shown in mesh). DNA models were taken from Figure~\ref{fig:Fig4.1} for sequences from -66 to +45. Asterisk in (B) indicates DNase I hypersensitive site at +3. Black lines in (B) indicate regions of continuous protection along SCP helix. }
\label{fig:Fig4.4}
\end{figure}
\indent The footprinting data provide new structural insight into the model of DNA through TFIID-TFIIA-SCP. Incorporating these footprinting data into the model reveals that the phasing of DNase I sensitive sites. On each strand between the Inr and MTE, there is a distinct 10 bp phasing of DNase I hypersensitive sites at -2, +18, and +28 (with probe DNA 5’-labeled upstream) and at +4 and +13 (with probe DNA 5’-labeled downstream) (Figure~\ref{fig:Fig4.1}A).  Incorporating this information into the cryo-EM model of TFIID-TFIIA-SCP allows for one side of the DNA sequence to face away from the central cavity (DNase I sensitive sites) while the opposite side of the helix faces the inner cavity of TFIID (DNase I protected sites) (Figure~\ref{fig:Fig4.4}). Hence, these data suggest that TFIID presents the DNase I-accessible face of this region of the core promoter to the bulk solution for interactions with RNAPII and other factors involved in transcription initiation. \\
\indent As the DNA extends across the central channel and enter into lobe A, the footprinting data suggest that there are topological changes in the promoter DNA surrounding the TSS. For instance, the DNase I footprinting experiment shows a hypersensitive site at +3, indicating that the DNA has changed conformed for optimal cleavage by Dnase I only when bound to TFIID (Figures~\ref{fig:Fig4.1}A). Given that the strength of this hypersensitive site appears to correlate with the strength of transcription initiation from a given promoter, it is interesting that this site is exposed within the central channel of TFIID, suggesting that TFIID-induced topological changes of DNA within the central channel are relevant for later steps in transcription initiation. \\
\indent The modeled path of promoter DNA through TFIID-TFIIA-SCP suggested that, in addition to the kinking of the TATA box by TBP, there is an additional bend in promoter DNA (Figure~\ref{fig:Fig4.4})). To test this model, MPE-Fe footprinting revealed that the intervening DNA between TATA and Inr showed sensitivty to MPE-Fe, indicating that the DNA helix is not distorted and there are not strong protein-DNA contacts. Therefore, while the DNA needs to bend within this area of the structure, it likely follows a gradual bending path as it enters into lobe A.\\

\section{TFIID interacts with diverse promoter architectures through the rearranged conformation}
\begin{figure}
\centering
\includegraphics[width=0.9\textwidth]{../Ch4_figs/Fig4.7.eps}
\caption[Core promoter architecture dictates TFIIA-dependent and TFIIA-independent interactions of TFIID with core promoter DNA]{Core promoter architecture dictates TFIIA-dependent and TFIIA-independent interactions of TFIID with core promoter DNA. DNase I footprinting on ‘wild type’ and ‘mutant’ SCP DNA sequences. 5’ labeled downstream probes were analyzed for DNase I protection in the presence or absence of TFIIA for wild type, mutant TATA (mTATA), Inr (mInr), and MTE/DPE (mMTE/DPE).}
\label{fig:Fig4.7}
\end{figure}
\begin{figure}
\centering
\includegraphics[width=0.8\textwidth]{../Ch4_figs/Fig4.8.eps}
\caption[Line profiles for DNase I footprinting on mTATA promoter]{Line profiles for DNase I footprinting on mTATA promoter.  (A) Line profiles for TFIID (blue) and TFIID-TFIIA (red) on ‘wild-type’ SCP.  (B) Line profiles for TFIID (blue) and TFIID-TFIIA (red) on mTATA promoter.}
\label{fig:Fig4.8}
\end{figure}
While the SCP DNA has served as an important tool for structurally dissecting the structure of TFIID-TFIIA-SCP, the presence of four consensus promoter motifs in SCP represents a non-physiological arrangement given that most human promoters contain only one or two consesus motifs \cite{Juven-Gershon_468}. To explore the effect of core promoter architecture on TFIID-promoter interactions, DNase I footprinting experiments were performed with ‘mutant’ SCP DNA constructs in the presence or absence of TFIIA (Figure~\ref{fig:Fig4.7}). Mutation of the TATA box within the SCP sequence (mTATA) resulted in a wild-type interaction with the promoter DNA from the Inr to the DPE, as seen previously \cite{Juven-Gershon_1249}. In addition, the inclusion of TFIIA resulted in a weak but detectable footprint over the mutant TATA box (Figure~\ref{fig:Fig4.8}). The strong resemblance between the mTATA and the wild-type SCP DNase I protection patterns suggested that TFIID is bound to the mTATA sequence in the rearranged conformation. To test this hypothesis, we collected cryo-EM data and visualized a sample of TFIID-TFIIA-SCP(mTATA) (Figure~\ref{fig:Fig4.6}). This experiment revealed that TFIID binds to the mTATA promoter in a nearly identical conformation as that observed with the wild-type SCP sequence. Thus, the combined footprinting and EM data indicate that the rearranged state of TFIID serves as the predominant DNA binding conformation for the SCP and mTATA promoter architecture.\\
\indent The conformation of promoter-bound TFIID was further addressed by analysis of promoters that contain mutations in the Inr (mInr) or MTE/DPE (mMTE/DPE) promoter motifs (Figure~\ref{fig:Fig4.7}). TFIID did not interact appreciably with either the mInr promoter or the mMTE/DPE promoter in the absence of TFIIA, as seen previously \cite{Juven-Gershon_1249}. However, the addition of TFIIA resulted in strong binding of TFIID to the TATA box as well as to sequences from the Inr through the DPE regions. In the presence of TFIIA, the overall patterns of protection observed with the mInr and mMTE/DPE promoters are similar to those seen with the wild-type SCP. It thus appears likely that TFIID-TFIIA binds to the mInr and mMTE/DPE promoters in the rearranged conformation.  Hence, the footprinting and EM data both suggest that TFIID binds to the wild-type and mutant SCPs in the newly discovered rearranged conformation.
\begin{figure}
\centering
\includegraphics[width=0.8\textwidth]{../Ch4_figs/Fig4.6.eps}
\caption[TFIID-TFIIA interacts with SCP(mTATA) within the rearranged conformation]{TFIID-TFIIA interacts with SCP(mTATA) within the rearranged conformation.  (A) 2D reference-free class averages calculated from cryo-EM data of TFIID-TFIIA-SCP(mTATA) are shown alongside 2D reference-free class averages from TFIID-TFIIA-SCP. (B) 3D models of the rearranged conformation for TFIID-TFIIA-SCP and TFIID-TFIIA-SCP(mTATA) at 34\AA. }
\label{fig:Fig4.6}
\end{figure}
\indent With the 'wild-type' SCP as well as with the three 'mutant' (mTATA, mInr, mMTE/DPE) versions of the SCP, we observed that TFIIA stimulates the binding of TFIID to the TATA box region (Figure~\ref{fig:Fig4.7}). This effect is consistent with the well-established TFIIA-mediated enhancement of TBP binding to the TATA box \cite{Thomas_1201}. With the mTATA promoter, the primary interaction of TFIID with the DNA is via the Inr, MTE, and DPE motifs, and a weak stimulation by TFIIA of the binding of TFIID to the mutant TATA box region is also observed. With the mInr and mMTE/DPE promoters, it seems likely that TFIIA stimulates the binding of TBP to the TATA box and that the remainder of the TFIID complex then interacts with the Inr through the DPE region of the core promoter, irrespective of the presence of consensus Inr or MTE/DPE elements. These findings may be analogous to the previously observed stimulation of the binding of partially-purified TFIID to the downstream promoter region of the adenovirus major late promoter (which lacks MTE/DPE motifs) by the upstream stimulatory factor, USF \cite{Sawadogo_3840,Va_3783}. In this light, it is possible that other sequence-specific activators, as well as coactivators, may function in a related manner to stabilize TFIID on promoter DNA and thus promote the formation of the rearranged conformation. 


\section{DNA binding within the canonical conformation}

While the rearranged conformation is the predominant form of TFIID bound to promoter DNA, we next wanted to investigate if the canonical state is capable of interacting with promoter DNA. Specifically, we wanted to know if the canonical state represented an 'inhibited' structural state of TFIID to bind the MTE/DPE, given the close proximity of lobe A and the MTE/DPE binding sites within the canonical conformation. This investigation into DNA binding by the canonical state is similar in analysis to the localization of TFIIA within lobe A (Figure~\ref{fig:Fig4.5}A), where 2D reference-free class averages of the canonical were generated and analyzed. \\
\indent After collecting larger datasets for both +45 and TATA-Nanogold labeled samples, class averages of the canonical conformation for TFIID-TFIIA-SCP showed that the promoter DNA binds in a manner similar to the rearranged conformation. This conclusion comes from comparison of the low defocus thresholded average (Figure~\ref{fig:Fig4.5}B, right) with the high defocus class averages obtained from alignment and classification (Figure~\ref{fig:Fig4.5}B, left). The presence of additional density extending away from lobe C in the high defocus average and the terminal localization of +45-Nanogold indicates that SCP DNA is bound. Furthermore, considering the characteristic extension of 15 bps away from lobe C within the rearranged conformation suggests that this additional density extending away from lobe C within the canonical conformation is promoter DNA bound in a near identical manner (Figure~\ref{fig:Fig4.5}). Unfortunately, the high degree of flexibility of lobe A did not allow careful comparison of lobe A's position and DNA binding, preventing conclusions from being drawn regarding lobe A inhibition of MTE/DPE binding. However, despite this limitation, analysis of the canonical conformation from TFIID-TFIIA-SCP(+45 gold) indicate that the canonical conformation is capable of binding MTE/DPE DNA sequences within the same configuration as the rearranged state.\\
\indent The binding of MTE/DPE by the canonical suggests that if lobe A engages the TATA box it should occur in a position located away from lobe C. Due to the rigid binding of the MTE/DPE by TFIID within both canonical and rearranged conformations, it is likely that the binding of the TATA box to lobe A occurs is a more flexible manner.  To test this, canonical class averages from TFIID-TFIIA-SCP(TATA gold) were analyzed for Nanogold localization (Figure~\ref{fig:Fig4.5}C). Unlike the stable binding of MTE/DPE to lobe C, the Nanogold signal for TATA gold was diffusey localized along the trajectory of lobe A's rearrangement from making contacts with lobe C (canonical) to lobe B (rearranged). Furthermore, considering the small cluster of peaks at a position opposite the central channel from lobe C, DNA binding in the canonical state positions the TATA box for binding by TBP within lobe A of the rearranged conformation (Figure~\ref{fig:Fig4.5}C). \\
\indent Analysis of the highly flexible canonical state using Nanogold labels for TFIIA and DNA has revealed insight into the dynamic process of DNA binding by TFIID (Figure~\ref{fig:Fig4.5}.  While it is difficult to know the position of lobe A within these Nanogold labeled class averages, the Nanogold labeling approach has provided key localizations in an otherwise flexible conformation. Despite the flexibility of lobe A, these labeling data indicate that the canonical conformation is competent for binding to promoter DNA. Lobe C binds to the MTE/DPE sequence in a similar fashion as the rearranged conformation, suggesting that its intrisinc DNA binding activity is preserved within the canonical state. Additionally, as suggested earlier, lobe A appears to exist as a modular TATA binding component of TFIID, where the TATA gold label localized to locations along the path of lobe A's rearrangement.\\ 
\begin{figure}
\centering
\includegraphics[width=1\textwidth]{../Ch4_figs/Fig4.5.eps}
\caption[Lobe C interacts with promoter DNA within the canonical conformation in a similar manner as the rearranged state]{Lobe C interacts with promoter DNA within the canonical conformation in a similar manner as the rearranged state. 2D reference-free class averages for the canonical conformation from cryo-EM data of TFIID-TFIIA-SCP with Nanogold labeled TFIIA (A), +45 (B), and TATA (C). Left of the averages is a model of the canonical state indicating location of Nanogold label. (D) Model of DNA path and TFIIA within the canonical conformation based upon gold labeling.}
\label{fig:Fig4.5}
\end{figure}

\section{TFIID-TFIIA-TFIIB-SCP adopts the rearranged conformation on promoter DNA}

The cryo-EM analysis presented thus far indicates that the predominant DNA binding form of TFIID is the rearranged conformation, suggesting that the remaining general transcription factors may load onto the rearranged state for RNAPII loading. To test this hypothesis, cryo-EM samples were prepared with TFIID-TFIIA-TFIIB-SCP(-66) and analyzed (Figure~\ref{fig:Fig4.9}. 2D reference-free class averages showed that the there were a variety of orientations containing DNA bound to the rearranged conformation (Figure~\ref{fig:Fig4.9}B). This suggests that the rearranged conformation is the conformation that is bound by TFIIB. \\
\begin{figure}
\centering
\includegraphics[width=1\textwidth]{../Ch4_figs/Fig4.9.eps}
\caption[Cryo-EM of TFIID-TFIIA-TFIIB-SCP(-66)]{Cryo-EM of TFIID-TFIIA-TFIIB-SCP(-66). Representative micrograph (A) and 2D reference-free class averages (B).  Scale bar is 200 nm in (A) and 200\AA in (B).}
\label{fig:Fig4.9}
\end{figure}
\indent The interaction of TFIIB with TFIID-TFIIA-SCP was further probed through DNase I and MPE-Fe footprinting of reactions containing TFIID, TFIIA, and TFIIB. The footprinting results show that TFIIB extends the length of the footprint previously around the TATA, while the remaining downstream contacts along the Inr, MTE, and DPE remain unchanged (Figure~\ref{fig:Fig4.11}). Given the bulky nature of DNase I vs. MPE-Fe as footprinting probes, the DNase I footprinting results indicate that an additional 8 - 10 bps are protected on the flanking regions of the TATA box, extending the footprint around the TATA box from -45 to -18 (Figure~\ref{fig:Fig4.11}A). This shows that nearly 30 bps of promoter DNA upstream of the TSS are sequested within the TFIID-TFIIA-TFIIB-SCP complex. Furthermore, given that the remaining downstream contacts with promoter DNA are unaffected by TFIIB, these data suggest that TFIIB preferentially interacts with the rearranged conformation. This result is consistent with the labeling of TFIIA and TATA box DNA within lobe A, suggesting that TFIIB binds to promoter DNA within lobe A.\\
\begin{figure}
\centering
\includegraphics[width=.7\textwidth]{../Ch4_figs/Fig4.11.eps}
\caption[TFIIB interacts with TFIID-TFIIA-SCP within the rearranged conformation on promoter DNA]{TFIIB interacts with TFIID-TFIIA-SCP within the rearranged conformation on promoter DNA. DNase I (A) and MPE-Fe (B) footprinting of TFIID-TFIIA-TFIIB-SCP on 5'-labeled downstream DNA. }
\label{fig:Fig4.11}
\end{figure}
\indent In addition to the DNase I footprinting experiments, MPE-Fe was used in order to provide high-resolution information on the protein-DNA contacts introduced by TFIIB within the TFIID-TFIIA-TFIIB-SCP complex. This analysis revealed that TFIIB makes strong contacts with the flanking DNA sequences around the TATA box from -34 to -19 (Figure~\ref{fig:Fig4.11}B). These results are consistent with the presence of a downstream BRE motif within the SCP sequence \cite{Juven-Gershon_1249}, a region of the promoter that is capable of engaging in sequence-specific contacts with TFIIB \cite{Deng_2005,Tsai_2000}. Interestingly, however, there are upstream contacts along the upstream BRE motif, even though the SCP sequence does not contain this element \cite{Juven-Gershon_1249}. Thus, TFIIB makes specific contacts with the upstream and downstream DNA sequences surrounding the TATA box, potentially stabilizing the TFIID-TFIIA-TFIIB-SCP complex due to an extended footprint on the DNA. \\
\begin{figure}
\centering
\includegraphics[width=.7\textwidth]{../Ch4_figs/4.12.eps}
\caption[TFIIB does not induce strong TATA box protection within TFIID-TFIIA-TFIIB-SCP(mTATA)]{TFIIB does not induce strong TATA box protection within TFIID-TFIIA-TFIIB-SCP(mTATA). DNase I footprinting of TFIID-TFIIA-TFIIB on SCP(mTATA). \emph{W} and \emph{M} indicate wild-type SCP and SCP(mTATA) promoters, respectively. }
\label{fig:Fig4.12}
\end{figure}
\indent The high affinity binding of TFIID-TFIIA-TFIIB-SCP to the TATA box and flanking sequences suggested that this complex may be able to bind to the mutant TATA sequence within SCP(mTATA). To test this hypothesis, we performed DNase I fooptrinting of TFIID-TFIIA-TFIIB on wild-type and mutant TATA box promoters (Figure~\ref{fig:Fig4.12}). The footprinting results revealed that the TATA, upstream BRE, and downstream BRE motifs were not longer strongly protected. Since TFIIB is unable to bind to the upstream and downstream BRE motifs in the absence of TFIIA (Figure~\ref{fig:Fig4.11}A), the footprinting on SCP(mTATA) suggests a cooperative activity of TFIIB-TBP-TATA in order for high affinity binding to the TATA, upstream BRE and downstream BRE motifs. This explanation is consistent with the crystal structure of TBP-TFIIB-TATA, where the TFIIB simultaneously contacts TBP, upstream BRE, and downstream BRE \cite{Tsai_2000}. It should be noted, however, that there is slight protection of the TATA sequence, comparable to that previously observed (Figure~\ref{fig:Fig4.8}), indicating that TBP may be within close proximity of the mutated TATA sequence. These footprinting data suggest that, within the context of a mutant TATA box, TFIIB does not stimulate DNA binding by TBP nor does it interact with the downstream BRE element, suggesting that TFIIB may be binding to the TFIID-TFIIA-SCP(mTATA) complex.\\
\indent The data above indicate that TFIIB binds the rearranged conformation in the presence of a functional TATA box and, given the extended footprinting surrounding the TATA box, the binding of TFIIB to TFIID-TFIIA-SCP may stabilize the ternary complex to enable high resolution studies. Therefore, 3D refinements were performed on a sample of TFIID-TFIIA-TFIIB-SCP(-66) using previously obtained models corresponding to the canonical and rearranged conformations. After collecting and analyzing 56,000 cryo-EM particles of TFIID-TFIIA-TFIIB-SCP, the resulting 3D refined model for the rearranged state achieved a resolution of 36\AA, comparable to that obtained previously for TFIID-TFIIA-SCP. To investigate if the binding of TFIIB to the ternary complex altered structural contacts between components within TFIID-TFIIA-SCP, the structure is shown alongside the previously obtained TFIID-TFIIA-SCP(-66) (Figure~\ref{fig:IIB_cryo}). The overall structural features of these two models are nearly the same at this resolution, indicating that TFIIB does not introduce global changes in the structure of TFIID-TFIIA-SCP upon binding. \\
\indent Despite these similarities, there are subtle changes of the DNA density within the TFIID-TFIIA-TFIIB-SCP(-66) structure. First, as a positive control, this sample was prepared with SCP(-66) to test the presence and location of the upstream DNA density that was identified previously (Figure~\ref{fig:Fig3.17}). Since the initial 3D model used for refinement did \emph{not} contain density corresponding to the upstream DNA, the presence of additional density exiting lobe A in a near-identical location suggests that position of upstream DNA does not change relative to the TATA box within TFIID-TFIIA-TFIIB-SCP(-66) (Figure~\ref{fig:IIB_cryo}). Interestingly, however, the shape of the DNA density appears to be narrower than the previously determined structure. This may be due to the increased rigidity of upstream DNA because of the TFIIB-mediated contacts along the flanking DNA sequences around the TATA box (Figure~\ref{fig:Fig4.11}A). Despite visualizing this change in upstream DNA, we were unable to detect any differences within lobe A (or any part of the TFIID structure) that is consistent in size with TFIIB (30 kDa). For clarity, the previously proposed atomic model of promoter DNA, TBP, TFIIA, and TFIIB were docked into the TFIID-TFIIA-TFIIB-SCP(-66) structure, which shows that the area of the proposed TFIIB binding site appears very similar between both structures. While TFIIB likely interacts with the rearranged conformation, the low resolution of the model prevents further conclusions to be drawn regarding an structural consequences of TFIIB binding. \\
\begin{figure}
\centering
\includegraphics[width=.8\textwidth]{../Ch4_figs/IID_IIA_IIB_SCP_fig.eps}
\caption[3D reconstruction of the rearranged conformation of TFIID-TFIIA-TFIIB-SCP(-66) reveals similar topology as TFIID-TFIIA-SCP(-66)]{3D reconstruction of the rearranged conformation of TFIID-TFIIA-TFIIB-SCP(-66) reveals similar topology as TFIID-TFIIA-SCP(-66). 3D models for rearranged conformation of TFIID-TFIIA-TFIIB-SCP(-66) (A) and TFIID-TFIIA-SCP (B) at 36\AA. (C) Docked DNA model for -66 to +45 with TBP-TFIIB-TFIIA crystal structure model (D) from Figure ~\ref{fig:Fig3.18} }
\label{fig:IIB_cryo}
\end{figure}

\section{A general model of regulated DNA binding by TFIID}
\begin{figure}
\centering
\includegraphics[width=0.8\textwidth]{../Ch4_figs/Fig4.13.eps}
\caption[Model for SCP DNA binding by TFIID-TFIIA]{Model for SCP DNA binding by TFIID-TFIIA. TFIID undergoes conformational changes between canonical (A) and rearranged (B) states. The addition of TFIIA stabilizes the canonical conformation (C). Alternatively, the addition of SCP DNA leads to binding of the Inr-MTE/DPE to the rearranged conformation (D).  The combined presence of TFIIA and SCP DNA leads to a stabilization of the rearranged conformation (G).  While most particles adopt the rearranged state for TFIID-TFIIA-SCP, there is a small subset that is bound to SCP DNA and TFIIA within the canonical state (E). This state converts to the rearranged state through a likely intermediate state (F). The rearranged TFIID-TFIIA-SCP conformation can then be bound by TFIIB to load RNAPII.}
\label{fig:Fig4.13}
\end{figure}

The structural work presented within this chapter has served to test and extend a model for promoter binding by TFIID. In order to model TFIID's interaction with promoter DNA, the model must take into account the interplay between TFIID's conformational dynamics and promoter binding. Therefore, the following models have been proposed in an attempt to summarize the structural data presented throughout Chapters 2, 3 \& 4 (Figures~\ref{fig:Fig4.13} - \ref{fig:Fig4.15}). While the rearranged state likely serves as the predominant high affinity DNA binding conformation of TFIID, the structural data presented here present alternative structural pathways for TFIID interacting with promoter DNA. Even though all of the experiments presented here were performed under equilibrium conditions, we believe that the equilibrium experiments implicitly inform us on the directionality of this strucural path taken by TFIID. The models that we proposed dependent on the interplay between core promoter architecture and TFIIA and, therefore, our models have been separated into based upon the core promoter architecture: SCP (Figure~\ref{fig:Fig4.13}), SCP(mTATA) (Figure~\ref{fig:Fig4.14}), and SCP(mMTE/DPE) (Figure~\ref{fig:Fig4.15}). Broadly, we believe these models serve as a framework for understading TFIID's pleiotropic interactions with promoter DNA in an activator-dependent manner.\\

\subsection{TFIID-TFIIA-SCP}

As previously suggested, TFIID's conformational state appears to be intimately connected to its promoter binding properities. Before binding DNA, however, TFIID already exhibits an unprecendented degree of conformational dynamics, where lobe A reoragnized between two distinct conformations: the canonical and rearranged states (Figures~\ref{fig:Fig4.13}A \& B). Measurements of lobe A position from both negative stain and cryo-EM sample preparations revealed that lobe A undergoes a reorganization between these two states. Surprisingly, these results indicated that approximately 50 of the TFIID molecules adopted the rearranged state before the addition of activators or promoter DNA  (Figures\ref{fig:Fig2.3}D; \ref{fig:Fig2.4}; \ref{fig:Fig2.5}E). Therefore, within our model, lobe A coexists equally between the canonical (Figure~\ref{fig:Fig4.13}A) and rearranged states (Figure~\ref{fig:Fig4.13}B). In order to explain the lobe A's pendulum-like motion, we have modeled lobe A to contain an unstructured linker domain that remains stably attached to the BC core throughout the process of lobe A's reorganization (Figure~\ref{fig:Fig4.13}A - B transition). \\
\indent From this existing equilibrium between the canonical and rearranged states, we propose that there are two alternative paths towards the high-affinity rearranged state bound to DNA. From analysis of lobe A positions, we found that lobe A equally adopts the canonical and rearranged states in the presence of SCP DNA (Figure~\ref{fig:Fig2.6}B). But, considering that the high affinity protection of Inr-MTE/DPE from DNase I and MPE-Fe probes in the presence and absensce of TFIIA (Figure~\ref{fig:Fig4.1}), we propose that TFIID interacts with SCP DNA within the rearranged conformation in the absense of TFIIA (Figure~\ref{fig:Fig4.13}D). Therefore, while the conformational dynamics of lobe A remain unchanged, the presence of high affinity Inr and MTE/DPE motifs within the SCP allows TFIID to interact with the DNA within the rearranged conformation.  Notably, TBP remains inhibited in the absence of TFIIA due to the absence of TATA protection (Figure~\ref{fig:Fig4.1}), which is likely due to the inhibitory N-terminal domain of TAF1 \cite{Bagby_2202,Geiger_2949,Liu_2574}. \\
\indent Alternatively, if TFIID is incubated with TFIIA alone in the absence of promoter DNA, lobe A is stabilized within the canonical conformation (Figure~\ref{fig:Fig4.13}C). Given the ability of TFIIA to stimulate high affinity TFIID-DNA complexes, we hypothesized that TFIIA should stabilize TFIID within the rearranegd state. Surprisingly, TFIIA stabilizes TFIID within the canonical state (Figure~\ref{fig:Fig2.6}C). While we do not understand the functional role of this canonical state-stabilization, we propose that this conformation primes TFIID for TBP-TATA interactions through the release of the inhibitory N-terminal domain of TAF1. Interestingly, re-analysis of activator-TFIID structures suggests that the transcriptional activators p53, c-Jun, and sp1 \emph{also} stabilize TFIID within the canonical state. Thus, there may be an additional regulatory role of the canonical state that remains unappreciated in our current studies.\\
\indent These two alternative branches of TFIID's conformational dynamics converge onto the rearranged conformation when TFIID is in the presence of TFIIA and SCP DNA. Given that a majority ( 60) of the particles for TFIID-TFIIA-SCP adopted the rearranged state (Figure~\ref{fig:Fig2.6}D) and were bound to promoter DNA (Figure~\ref{fig:Fig2.7}A), we modeled the conformational landscape of TFIID-TFIIA-SCP to faciliate the formation of the rearranged state (Figure~\ref{fig:Fig4.13}G). As previously mentioned, we believe that the transition from (Figure~\ref{fig:Fig4.13}D) to (Figure~\ref{fig:Fig4.13}G) involves the stabilization of TBP-TATA interactions through a release of TBP inhibition. This is supported by the strong TATA box protection observed  TFIID-SCP only in the presence of TFIIA. On the other hand, we believe that there are at least two structural transitions that are necessary for TFIID-TFIIA when SCP is added (Figure~\ref{fig:Fig4.13}E, \& F). Through the use of gold labeling, we localized TFIIA and SCP DNA to the canonical state for TFIID-TFIIA-SCP (Figures~\ref{fig:Fig3.16}D \& \ref{fig:Fig4.5}). While TFIIA was localized to lobe A, the SCP DNA was oriented in a manner that the MTE and DPE motifs were bound by lobe C (Figure~\ref{fig:Fig4.5}B), in a similar conformation as the MTE/DPE binding within the rearranged conformation. Therefore, we have modeled the canonical state for TFIID-TFIIA-SCP to contain lobe C interacting with the MTE/DPE in addition to the TFIIA-mediated release of TAF1's inhibitory domain from TBP (Figure~\ref{fig:Fig4.13}E). This conformation is primed for rearrangement due to the availability of TBP for binding the TATA box, where the organization of SCP DNA drives the conformational landscape of TFIID to the rearranged state given the binding of MTE/DPE to lobe C and TATA/Inr to lobe A. Given the flexible nature of lobe A's attachment to the BC core, the structural reorganization between the canonical and rearranged states likely involves an intermediate state, where DNA is bound to lobe C and TFIIA is within lobe A (Figure~\ref{fig:Fig4.13}F). The culmination of these processes results in the net-stabilization of TFIID-TFIIA-SCP within the rearranged state (Figure~\ref{fig:Fig4.13}G).\\ 
\indent While there still remains limited information the structural organization TFIID's subunits, the path of DNA through the TFIID-TFIIA complex can serve as a label for subunit locations. Through DNA cross-linking experiments, the MTE and DPE were found to crosslink to TAF6 and TAF9 \cite{Burke_2739,Lim_1522}. In addition to these experiments, reconstitution of a TAF6, -9, -4, -12 subcomplex was sufficient to interact with DPE-containing promoters \emph{in vitro} \cite{Shao_1340}. Therefore, we modeled lobe C to mediate the contacts between TFIID and the MTE/DPE region of the SCP (Figure~\ref{fig:Fig4.13}).\\
\indent The composition of lobe A has likely contains TBP, TAF1, and TAF2, serving as a modular sub-complex within TFIID. 
\indent To provide independent support for these conclusions, prevoiusly obtained TFIID-antibody datasets were extensively analyzed through the lobe A analysis for antibodies directed against TBP, TAF1, TAF6, and TAF4.   estTAF6 labeling?

\subsection{TFIID-TFIIA-SCP(mTATA)}

\subsection{TFIID-TFIIA-SCP(mMTE/DPE)} 

\begin{figure}
\centering
\includegraphics[width=0.8\textwidth]{../Ch4_figs/Fig4.14.eps}
\caption[Model for SCP(mTATA) DNA binding by TFIID-TFIIA]{Model for SCP(mTATA) DNA binding by TFIID-TFIIA. TFIID undergoes conformational changes between canonical (A) and rearranged (B) states. The addition of SCP(mTATA) interacts with the rearranged conformation (C) that is further stabilized by TFIIA (D). Note that the DNA likely makes weak contacts with the surface of TFIID near TBP without TBP engagement of DNA. The rearranged TFIID-TFIIA-SCP(mTATA) conformation should then be bound by TFIIB, although it is difficult to know where within the TFIID structure it will bind (E).}
\label{fig:Fig4.14}
\end{figure}

\begin{figure}
\centering
\includegraphics[width=0.8\textwidth]{../Ch4_figs/Fig4.15.eps}
\caption[Model for SCP(mMTE/DPE) DNA binding by TFIID-TFIIA]{Model for SCP(mMTE/DPE) DNA binding by TFIID-TFIIA. TFIID undergoes conformational changes between canonical (A) and rearranged (B) states. The addition of TFIIA stabilizes the canonical conformation (C). The addition of SCP(mMPE/DPE) leads to the formation of the rearranged conformation (F) through the proposed intermediates of (D) and (E). The rearranged TFIID-TFIIA-SCP(mMTE/DPE) conformation can then be bound by TFIIB to load RNAPII (G).}
\label{fig:Fig4.15}
\end{figure}

\begin{figure}
\centering
\includegraphics[width=0.8\textwidth]{../Ch4_figs/BentDNA_mdoels.eps}
\caption[TFIID may facilitate transcription initiation through topological changes in promoter DNA]{TFIID may facilitate transcription initiation through topological changes in promoter DNA. Model of promoter DNA path through TFIID-TFIIA-SCP(-66) for core promoter element location (A), DNase I footprinting (B), and MPE-Fe footprinting (C). Shown alongside the modeled DNA path through an elongating RNAPII (adopted from PDB 3PO3). (E) Corresponding view from TFIID-TFIIA-SCP(-66) model.}
\label{fig:BentDNA}
\end{figure}
