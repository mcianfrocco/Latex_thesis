\chapter{Human TFIID binds promoter DNA in a reorganized structural state}

\section{Detailed analysis of TFIID reveals dramatic lobe A flexibility}

\subsection{Lobe A is flexibily attached to a stable core of TFIID that comprises lobes B and C} 
\begin{figure}
\centering
\includegraphics[width=1\textwidth]{../Ch2_figs/Fig2.1.eps}
\caption[Micrographs of negatively stained and vitrified TFIID samples]{Micrographs of negatively stained and vitrified TFIID samples. (A) Negatively stained sample of TFIID in uranyl formate.  (B) Cryo-EM micrograph of TFIID.  Representative particles are shown in red boxes.  Scale bars are 100 nm.} 
\label{fig:Fig2.1}
\end{figure}
In a previous cryo-EM analysis of human TFIID, the use of a 3D variance approach indicated that conformational flexibility is an intrinsic property of TFIID \cite{Grob_1281}. Due to the potential role of this flexibility in TFIID function, we investigated this property more thoroughly by extensive 2D image analysis of negatively stained TFIID samples (Figure~\ref{fig:Fig2.1}A and Figure~\ref{fig:Fig2.2}A). The resulting 2D averages obtained showed a variety of different view of TFIID, as seen previously \cite{Grob_1281}. However, upon careful analysis, we noticed that a number of the class averages that initially looked like different views of TFIID actually corresponded to the same orientation (Figure~\ref{fig:Fig2.2}B). When comparing these averages side-by-side, it becomes clear that there is a common rigid sub-structure within TFIID that is shared between all of the class averages. Surprisingly, a flexibility attached lobe is connected to the rigid sub-structure in a variety of conformations (Figure~\ref{fig:Fig2.2}B). Thus, these class averages indicate that TFIID may exhibit a greater degree of flexibility than previously appreciated.\\
\begin{figure}
\centering
\includegraphics[width=1\textwidth]{../Ch2_figs/Fig2.2.eps}
\caption[2D reference-free class averges of negatively stained TFIID]{2D reference-free class averges of negatively stained TFIID. (A) Representative class averages obtained during analysis of all particles within dataset. (B) Selected averages from (A) that correspond to a similar particle orientation to each other. Scale bars are 200\AA\.} 
\label{fig:Fig2.2}
\end{figure}
\indent Comparison of these class averages with a 3D model indciate that the flexible domain is lobe A. Alignment of the apo-TFIID 3D model (Figure~\ref{fig:Fig2.3}A \& B) with these averages (Figure~\ref{fig:Fig2.2}B) shows that the 3D model appears to be able to align the different averages within different projections (Figure~\ref{fig:Fig2.3}C). However, upon closer investigation, the rigid sub-structure within the projections is seen to 'shrink' between the first (top row, left) and last (bottom row, right) class average (Figure~\ref{fig:Fig2.3}C). To improve the alignment between the 3D model and 2D class averages, only the rigid sub-structure of the model was aligned (Figure~\ref{fig:Fig2.3}B \& D). This new model aligned all of the averages within the same orientation, confirming that these class averages correspond to specific structural states within a single orientation of TFIID. Furthermore, this analysis indicates that lobe A is the flexible lobe A that differs in position relative to the rigid sub-substructure that comprises lobes B and C, termed 'BC core.' \\
\indent To confirm that the conformational flexiblity seen for lobe A was not an artifact due to negatve staining procedures, cryo-EM data for TFIID were analysis in an identical manner. Indeed, after performing 2D reference-free alignment on the particles, similar conformational states were observed within the cryo-EM for TFIID where lobe A is flexibly attached to a stable BC core (Figure~\ref{fig:Fig2.4}). Thus, lobe A undergoes a ~ 100\AA\ structural reorganization as lobe A changes connectivity from lobe C to lobe B. Given that the structural state where lobe A contacts lobe C corresponds to the previously determined 3D structure of TFIID \cite{Andel_2407,Grob_1281}, we will refer to this state as the 'canonical' state. When lobe A contacts lobe B, TFIID forms a rearranged horseshoe structure, hence we will refer to the newly discovered structure as the 'rearranged' state (Figure~\ref{fig:Fig2.4}).\\
\begin{figure}
\centering
\includegraphics[width=1\textwidth]{../Ch2_figs/Fig2.3.eps}
\caption[Lobe A exists in a range of positions relative to a stable BC core within TFIID]{ Lobe A exists in a range of positions relative to a stable BC core within TFIID. (A \& B) Negative stain reconstruction of TFIID from previous work \cite{Grob_1281}.  Lobe positions are indicated. (C) Class averages from Figure 2.2B aligned to reprojections of model in (A). (D) Class averages from Figure 2.2B aligned to reprojections of TFIID’s BC core. The BC core is indicated in (A) by the dotted line. Scale bars are 100\AA\ .} 
\label{fig:Fig2.3}
\end{figure}
\begin{figure}
\centering
\includegraphics[width=0.6\textwidth]{../Ch2_figs/Fig2.4.eps}
\caption[Cryo-EM class averages of TFIID confirm lobe A’s flexibility]{Cryo-EM class averages of TFIID confirm lobe A’s flexibility. Selected 2D reference-free class averages of TFIID highlight large movement of A relative to a stable BC core.  ‘Canonical’ and ‘rearranged’ conformationals states are indicated. The BC core is colored in blue while lobe A is colored in yellow. Scale bar is 200\AA\ .}
\label{fig:Fig2.4}
\end{figure}
\section{Focused classification \& measurement of lobe A position describes TFIID's conformational landscape}
After characterizing the extent of lobe A's conformational variability, we wanted to quantify the relative occupancy of each structural state to arrive at a more complete description of TFIID's conformational landscape. To this end, we concentrated on a “standard” view, where the three main lobes of TFIID, lobes A, B and C, were clearly separated. After performing a 2D reference-free alignment, all particles corresponding to this standard view were extracted and re-aligned to a reference of the BC core only (Figure~\ref{fig:Fig2.5}A). The average of these re-aligned particles showed that lobe A was 'blurred-out' due to wide range of conformational states that lobe A adopts (Figure~\ref{fig:Fig2.5}A). Subclassification within a 2D mask that excluded the stable BC core revealed that lobe A adopts a wide range of positions (Figures~\ref{fig:Fig2.5}B), as seen previously (Figure~\ref{fig:Fig2.2}B). However, unlike the previously obtained averages, this analysis provides a direct comparison of the occupancy of each structural state. \\
\begin{figure}
\centering
\includegraphics[width=1\textwidth]{../Ch2_figs/Fig2.5.eps}
\caption[Focused classification and positional measurements of lobe A within TFIID]{Focused classification and positional measurements of lobe A within TFIID. (A) Strategy for alignment \& classification.  Particles were selected based upon membership within characteristic 2D reference-free averages that were calculated from all particles within a given dataset.  These selected particles were then aligned to the rigid BC core and classified within a mask centered on lobe A.  (B) Resulting subclassified classums from negatively stained TFIID particles after hierarchical ascendant classification.  (C) Resulting subclassified classums from cryo-EM of TFIID particles after hierarchical ascendant classification.  (D) Schematic for measuring lobe A position within subclassified classums.  The position of lobe A was projected onto the B-C axis to provide a measurement for lobe A’s localization within TFIID. (E) Histogram of lobe A positions for negatively stained (left) and vitrified TFIID (right).  Scale bar in (A) is 200\AA\ .}
\label{fig:Fig2.5}
\end{figure}
\indent To quantify the structural plasticity exhibited by TFIID, the position of lobe A along the BC core was measured within each sub-classified average. The measurements were normalized so that values greater than 0.70 correspond to particles resembling the canonical state, in which lobe A is at its closest position to lobe C (Figure~\ref{fig:Fig2.5}D). Values less than 0.50, on the other hand, indicate that the particles are in a rearranged state, wherein lobe A is proximal to lobe B (Figure~\ref{fig:Fig2.5}D). This analysis revealed that the position of lobe A can be described by a bi-modal distribution with peaks centered at 0.40 and 0.80 (Figure~\ref{fig:Fig2.5}E and Figure~\ref{fig:Fig2.6}A). Surprisingly, this analysis showed that approximately 50\% of the TFIID particles are found in the rearranged state. These distinct structural states of TFIID were observed in datasets from both cryo-EM and negative stain data, demonstrating that the sample preparation did not alter the results (Figures~\ref{fig:Fig2.5}C \& E). Hence, there are two predominant and structurally distinct states of TFIID that differ the reorganization of lobe A within the complex.
\begin{figure}
\centering
\includegraphics[width=.75\textwidth]{../Ch2_figs/Fig2.6.eps}
\caption[TFIID’s conformational landscape changes in response to TFIIA and SCP DNA]{TFIID’s conformational landscape changes in response to TFIIA and SCP DNA. Distribution of lobe A positions relative to the stable BC core for TFIID (A), TFIID-SCP (B), TFIID-TFIIA (C), and TFIID-TFIIA-SCP samples (D). Inset within (A): Class averages corresponding to specific lobe A measurements from the TFIID sample.}
\label{fig:Fig2.6}
\end{figure}
\section{TFIIA and SCP DNA modulate the position of lobe A within TFIID}
\subsection{The combined presence of TFIIA and SCP DNA stabilize the rearranged state}
To examine whether the binding of TFIIA and promoter DNA is linked to the conformational states of TFIID, we analyzed TFIID in the presence of excess super core promoter (SCP) DNA and/or the cofactor TFIIA. The SCP contains the TATA box, Inr, MTE, and DPE core promoter motifs (Figure~\ref{fig:Fig1.4}). When TFIID was incubated with SCP DNA and negatively stained samples were visualized and analyzed as described above, lobe A showed a similar distribution of positions relative to TFIID (Figure~\ref{fig:Fig2.6}B) (p = 0.0353, Wilcoxon signed-rank test). This result suggested either that TFIID does not interact with SCP under the conditions used, or that DNA binding does not alter the conformational partitioning of lobe A within the accuracy of our measurements.\\
\indent We then tested whether the presence of TFIIA affects the properties of TFIID. When TFIID was incubated with TFIIA in the absence of DNA, the distribution of lobe A positions was shifted to the peak centered at 0.80. It thus appears that TFIIA stabilizes TFIID in the canonical state  (Figure~\ref{fig:Fig2.6}C). On the other hand, in the presence of SCP DNA, TFIIA promotes the formation of the rearranged state of TFIID, with a strong increase in the peak of lobe A positions centered at 0.40  (Figure~\ref{fig:Fig2.6}D). The distribution of lobe A positions within the context of TFIID-TFIIA  (Figure~\ref{fig:Fig2.6}C) was significantly different from that seen with TFIID-TFIIA-SCP  (Figure~\ref{fig:Fig2.6}D) (p = 1.5 x 10-13, Wilcoxon signed-rank test). Since TFIID is purified using established protocols \cite{Liu_723,Liu_574}, any underlying biochemical heterogeneity will be constant between these experiments, suggesting that these conformational changes are due to the presence of TFIIA and SCP DNA. These results indicate that TFIIA serves the dual function of maintaining TFIID in the canonical state in the absence of DNA and promoting the formation of the rearranged state in the presence of promoter DNA.\\

\subsection{Cryo-EM analysis verified the presence of DNA within rearranged state of TFIID}
To characterize the structure of this novel rearranged state of TFIID and its binding to DNA, we carried out cryo-EM visualization of frozen hydrated samples. Analysis of 2D reference-free averages from cryo-EM data of the TFIID-TFIIA-SCP ternary complex showed the presence of both the rearranged and canonical states (Figures~\ref{fig:Fig2.7}A \& B, leftmost panels). Importantly, the class averages corresponding to the rearranged state revealed extra density that appeared to be DNA extending across the central channel of TFIID (Figures~\ref{fig:Fig2.7}A, left panel). This additional density was confirmed to be DNA by examining cryo-EM 2D reference-free class averages collected for TFIID, TFIID-SCP, and TFIID-TFIIA. 2D reference-free class averages of TFIID without the combined presence of TFIIA and SCP showed no extra density within the rearranged state (Figure~\ref{fig:Fig2.7}A). These observations suggest that TFIIA is required for efficient binding of TFIID to SCP and that the predominant form of TFIID that is bound to DNA is the novel rearranged state.

\section{Discussion}
Even though conformational flexibility was orignially observed for lobe A within frozen-hydrated samples of TFIID \cite{Grob_1281}, this surprising degree of flexibility for lobe A went unappreciated for the past 10 years of research on human TFIID \cite{Andel_2407,Grob_1281,Liu_723,Liu_574}. Furthemore, re-analysis of all datasets of TFIID available reveal the presence of the rearranged conformation, in addition to the continuum of conformational states adopted by lobe A (data not shown). This indicates that the rearranged conformation has been consistently overlooked (e.g. Supplemental Figure S4 in \cite{Liu_723}). Given the subtle differences observed between projections of the canonical state when it was aligned to 2D class averages of the rearranged state (Figure~\ref{fig:Fig2.3}C), it is not surprsing that this slight difference was overlooked.\\
\begin{figure}
\centering
\includegraphics[width=0.7\textwidth]{../Ch2_figs/Fig2.7.eps}
\caption[Cryo-EM analysis confirms that the rearranged conformation of TFIID interacts with SCP promoter DNA in the presence of TFIIA]{Cryo-EM analysis confirms that the rearranged conformation of TFIID interacts with SCP promoter DNA in the presence of TFIIA. 2D reference-free class averages obtained for the rearranged conformationa (A) and the canonical conformation (B) in the presence of the indicated factors.  Note the presence of linear density within the rearranged conformation for TFIID-TFIIA-SCP.  Scale bar is 200\AA\ .}
\label{fig:Fig2.7}
\end{figure}
\indent The pendulum nature of the lobe A's movement relative to the stable BC core suggests that lobe A maintains a connection to the BC core within all conformational states. The nature of this linkage remains unknown and, given that there is not visible extra density connecting lobe A to the BC core, we postulate that the linking domain is an unstructured polypeptide chain. Futhermore, given the lack of information on subunit organization within TFIID, it is difficult to speculate on the composition of lobe A or the BC core at this time (see Chapters 3 \& 4 for model of TFIID composition). Given the large distance (\textgreater 100 \AA\ )that lobe A moves relative to the BC core, understanding the molecular basis for modulation of lobe A's position may have important implications for regulating TFIID activity.   \\ 
