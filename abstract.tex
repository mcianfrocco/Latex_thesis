\begin{abstract}

Proper gene regulation is a problem faced by all organisms, ranging from single-celled bacteria to multi-cellular mammals. In order to respond to the changing environment, thousands of genes across the genome must controlled in a coordinated manner. The process of evolution has placed these regulatory networks under constant selective pressure, which has resulted in the intricate assembly of protein-protein, protein-RNA, and protein-DNA interactions necessary for accurate gene expression. Considering that the mis-regulation of genes results in human diseases that range from autoimmune disorders to cancers, it has been a major focus of modern molecular biology to understand the molecular mechanisms that determine these complex patterns of gene expression. Advancing our understanding of these processes will illuminate the process of evolution in addition to providing better tools to diagnose and treat human diseases.\\
\indent In order to study the molecular underpinings of transcription regulation in multicellular organisms, we have focused our investigation on the key transcription factor known as TFIID. As a 13-14 protein complex, TFIID serves as an important regulatory hub during the process of transcription initiation by simultaneously interacting with distal activators and repressors, promoter DNA elements, and the basal transcription machinery. TFIID integrates these signalling cues to initiate RNAPII loading at specific genes across the genome, thus ensuring the survival of the cell and organism as a whole. Considering that TFIID is a conserved throughout eukaryotic life, understanding its ability to license RNAPII transcription stands to provide deep insight into gene regulation across many clades of life.  \\
\indent To address the structural basis for TFIID's ability to communicate with upstream activators and promoter DNA elements, we used single particle electron microscopy to visualize human TFIID bound to TFIIA and promoter DNA. This analysis revealed that TFIID co-exists in two predominant and distinct structural states that differ by a 100\AA\ translocation of TFIID’s lobe A. The transition between these structural states is modulated by the activator TFIIA, as the presence of TFIIA and promoter DNA facilitates the formation of a novel rearranged state of TFIID capable of promoter recognition and binding. DNA-labeling and footprinting, together with cryo-EM studies, mapped the locations of the TATA, Inr, MTE, and DPE promoter motifs within the TFIID-TFIIA-DNA structure. The existence of two structurally and functionally distinct forms of TFIID suggests that the different conformers may serve as specific targets for the action of regulatory factors.

\end{abstract}
