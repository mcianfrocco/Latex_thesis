\chapter{Cryo-EM analysis of TFIID-TFIIA-SCP defines the 3D organization of DNA and TFIIA bound to TFIID}

\section{Implementation of OTR to generate model of the rearranged state}

\subsection{Principles of \emph{ab initio} 3D reconstruction strategies}

The refinement and 3D reconstruction of individual particles in single particle EM requires an initial model to be input by the user. Unfortunately, given that the single particles are inherently noisy due to low-dose image collection strategies, 3D reconstructions from single particle datasets can be sensitive to the initial model originally provided. Furthermore, given the 'model bias' problem observed for aligning individual noisy particles with a reference projection \cite{Stewart_2004}, generating an initial model for the refinement of protein structures represents a non-trivial step in single particle EM. \\
\indent There are three commonly used approaches for calculating \emph{ab initio} 3D models for use as initial models for 3D refinements of single particles: angular reconstitution \cite{VanHeel_1987}, RCT \cite{Radermacher_1987}, and OTR  \cite{Leschziner_1228}. Each method relies on calculating 2D reference-free class averages through an iterative process of classification and alignment. From these averages, angular reconstitution relies upon the shared 'common line' in Fourier space between different 2D views of the same object. The 'common line' is the direct result from the central section theorem of image formation within transmission EM \cite{DeRosier_1968}. By calculating the 'common lines' between 2D reference-free class averages, an \emph{ab initio} 3D reconstruction can be obtained for the particles analyzed. While this technique provides a powerful tool for calculating 3D models of both symmetric and asymmetric structures, it is limited by the assumption that the 2D reference-free averages represent different view of the \emph{same} object. Therefore, conformational or biochemical heterogeneity within the sample can result in 2D class averages that are \emph{not} from the same object and, when these averages are combined during angular reconstitution, the resulting 3D model will not be an accurate representation of the underlying protein structure.\\
\begin{figure}
\centering
\includegraphics[width=1\textwidth]{../Ch3_figs/Fig3.1.eps}
\caption[Principles of OTR and RCT]{Principles of OTR and RCT. (A) First, the stage is tilted by +45\textdegree\ in the microscope before taking the first exposure. After recording the exposure, the stage is then tilted to -45\textdegree\ and a second exposure is taken. Superposition of a tilt pair reveals that the images are separated by 90\textdegree\ . (B) Another particle tilt pair is collected and, when averaged together with the first image through in-plane rotation of the +45\textdegree\ image, the particles further fill-in the 3D structure of the class average.  (C) After averaging many particles that have been aligned through in-plane rotations, 3D information is built up for a given class average until a 3D reconstruction can be performed. Note that RCT is identical to the description of OTR here with the exception that images are collected at 0\textdegree and 60\textdegree.}    
\label{fig:Fig3.1}
\end{figure}
\indent As an alternative methodology to angular reconstitution, RCT relies on collecting particle tilt pairs at a defined angle in order to calculate \emph{ab initio} 3D reconstructions \cite{Radermacher_1987} (Figure~\ref{fig:Fig3.1}). Since each particle collected for RCT has a corresponding tilt mate at known tilt angles (60 - 65\textdegree), generating 2D class averages for the untilted particles results in grouping tilt mates from different views together. And, since the angles between all of the particle tilt mates are known, 3D reconstructions can be calculated for each class average. This approach for calculating \emph{ab initio} 3D reconstructions does not make any assumptions about the sample homogeneity (unlike angular reconstitution, see above), instead relying on the geometry of particle tilt pairs to provide structural information on 2D class averages. The only limitation with this technique is that the maximum tilt angle achievable in the EM is 60 - 65\textdegree\ due to the physical nature of the sample holder and support grid. This results in the famous 'missing-cone problem,' where there is a large cone-shaped area in reciprocal space that does not contain any structural information and can lead to artifacts within 3D reconstructions \cite{Frank_1996}.\\ 
\indent  These limitations of RCT led to the recent development of OTR as an \emph{ab initio} methodology that does not suffer from the 'missing-cone problem' (Figure~\ref{fig:Fig3.1}) \cite{Leschziner_452,Leschziner_1228}. The conceptual framework of OTR is similar to RCT: particle tilt pairs are collected in the EM through direct manipulation of the stage angle, but, instead of collecting images at 0\textdegree and 60\textdegree as in RCT, OTR images are collected at +/- 45\textdegree. This technique has the ability to fill in completely all of reciprocal space for a given 2D class average (Figure~\ref{fig:Fig3.1}). The full description of 3D structural information for OTR class volumes abolishes the requirement for sub-volume averaging techniques, a technique normally used for merging RCT class volumes to remove the 'missing-cone problem.' While there are robust techniques for accurate merging of RCT class volumes \cite{Scheres_2009}, combining class volumes of pseudo-symmetric volumes (like TFIID, see below) may result in loss of 3D structural information of distinct structural states. \\

\subsection{3D reconstruction and refinement of rearranged conformation}
\begin{figure}
\centering
\includegraphics[width=1\textwidth]{../Ch3_figs/Fig3.2.eps}
\caption[OTR tilt pair micrographs for TFIID-TFIIA-SCP]{OTR tilt pair micrographs for TFIID-TFIIA-SCP. Scale bar is 200 nm.}
\label{fig:Fig3.2}
\end{figure}
In order to calculate an \emph{ab initio} 3D model for the rearranged conformation, the above points were considered in deciding to implement OTR in determining the 3D structure of TFIID-TFIIA-SCP (Figures~\ref{fig:Fig3.2} \& \ref{fig:Fig3.3}). Since OTR preserves three-dimensional features of individual class volumes without suffering from the 'missing-cone' problem of RCT \cite{Chandramouli_280}, individual class volumes calculated from 2D reference-free class averages of TFIID-TFIIA-SCP (Figure~\ref{fig:Fig3.3}A) were refined against untilted negative stain data for TFIID-TFIIA-SCP.  These refined models were then quantitatively compared to the characteristic class averages of TFIID in six distinct conformations (Figure~\ref{fig:Fig3.3}C \& D) and models were excluded based upon Fourier Ring Correlation (FRC) \cite{Saxton_3960} values below 60\AA\ (Figure~\ref{fig:Fig3.3}D).  A majority (8/10) of the refined, validated models were found to be in the rearranged state (Figure~\ref{fig:Fig3.3}E, c - g), whereas a minority (2/10) were in the canonical conformation (Figure~\ref{fig:Fig3.3}E, a).  \\
\indent Comparison of the resulting refined 3D models corresponding to the canonical and rearranged conformations revealed the three-dimensional path of lobe A's rearrangement (Figure~\ref{fig:Fig3.4}). As expected from analysis of 2D reference-free averages of TFIID (Figure~\ref{fig:Fig2.2}B), lobe A's movement from the canonical conformation, where it contacts lobe C, to the rearranged conformation, contacting lobe B, mostly involves a translational component on the BC core (Figure~\ref{fig:Fig3.4}A \& B). Additionally, the BC core appears to be nearly identical between the two structures, indicating that a majority of the structural changes exhibited by TFIID occur through the repositioning of lobe A (Figure~\ref{fig:Fig3.4}A \& B, blue). While it is difficult to orient lobe A between the two conformations, lobe A likely rotates since the connecting density from A to C in the canonical state may rotate to contact lobe B in the rearranged conformation (Figure~\ref{fig:Fig3.4}A \& B, arrows). Unfortunately, despite the structural information gathered from negative stain on TFIID-TFIIA-SCP, nucleic acids are not preserved within the negative stain. This required that cryo-EM sample preparation and analysis was necessary to visualize the path of DNA through TFIID-TFIIA-SCP.\\ 
\begin{figure}
\centering
\includegraphics[width=1\textwidth]{../Ch3_figs/Fig3.3.eps}
\caption[OTR implementation and model validation for both canonical and rearranged reconstructions of TFIID-TFIIA-SCP]{OTR implementation and model validation for both canonical and rearranged reconstructions of TFIID-TFIIA-SCP. (A) 2D reference-free class averages for -45\textdegree\ tilt data.  (B) Strategy for validating OTR models.  (C) Untilted TFIID-TFIIA-SCP 2D reference-free class averages that were used to assess model quality for refined OTR class volumes. (D) Highest \& lowest resolution models shown as the top two and bottom two rows, respectively.  (E) 3D model projections for refined \& validated OTR models.  Only 5 out of 8 rearranged models (c - g) and 1 out of 2 (a) canonical models are shown for space considerations.  For comparison, the corresponding projection for a previously determined TFIID negative stain structure of the canonical state is shown (b) \cite{Grob_1281}.}
\label{fig:Fig3.3}
\end{figure}
\begin{figure}
\centering
\includegraphics[width=.8\textwidth]{../Ch3_figs/Fig3.4.eps}
\caption[3D reconstructions of TFIID-TFIIA-SCP in canonical and rearranged conformations from negatively stained particles]{3D reconstructions of TFIID-TFIIA-SCP in canonical and rearranged conformations from negatively stained particles.  Canonical (A) and rearranged (B) reconstructions at 30\AA\ and 32\AA , respectively. The stable BC core is colored in blue in both structures while lobe A is colored orange within the canonical conformation and yellow in the rearranged conformation. 2D reference-free averages were aligned to the models in a multi-model manner, where the best matching averages for the canonical (C) and rearranged (D) conformation are shown.  Scale bars are 200\AA\ .}
\label{fig:Fig3.4}
\end{figure}

\section{Comparison of TFIID-TFIIA-SCP cryo-EM structures in the canonical and rearranged conformations}

Previous 2D image analysis indicated that both TFIID and the ternary TFIID-TFIIA-SCP samples existed in a distribution of conformational states (Figure~\ref{fig:Fig2.6}A \& D), requiring the use of a multi-model 3D refinement strategy. A total of ~35,000 cryo-EM particle images from the purified TFIID-TFIIA-SCP complex were sorted using multi-reference projection matching with two distinct reference structures that represent the canonical and rearranged conformations. After implementing a cross-correlation cut-off that excluded 25\% of the particles, the reconstruction of the rearranged state (comprising 60\% of the remaining particles) was refined to a resolution of 32\AA\ (Figure~\ref{fig:Fig3.5}C). A prominent feature of the rearranged structure is the presence of density (Figure~\ref{fig:Fig3.5}C, green), which we attribute to DNA, extending over lobe C and across the central channel of TFIID towards the region connecting lobes A and B. \\
\begin{figure}
\centering
\includegraphics[width=0.94\textwidth]{../Ch3_figs/Fig3.5.eps}
\caption[TFIIA-mediated binding of SCP DNA to the rearranged state of TFIID]{TFIIA-mediated binding of SCP DNA to the rearranged state of TFIID. 3D reconstructions of the canonical conformation for TFIID-TFIIA-SCP at 32\AA\ (A) and TFIID at 35\AA\ (B).  3D reconstructions of the rearranged conformation for TFIID-TFIIA-SCP at 32\AA\ (C) and TFIID at 35\AA\ (D), where density in (C) attributed to DNA is shown in green. The stable BC core is colored blue for each model. Lobe A is colored orange for (A) and (B) and yellow for (C) and (D). 2D reference-free averages for TFIID-TFIIA-SCP (E \& F) and TFIID (G \& H) are aligned to canonical and rearranged conformations, respectively. Scale bars are 200\AA\ .}
\label{fig:Fig3.5}
\end{figure}
\indent A 3D model for the canonical state was refined simultaneously from the TFIID-TFIIA-SCP samples to a similar resolution and included the remaining 40\% of the selected particles (Figure~\ref{fig:Fig3.5}A). Importantly, the canonical state does not show the presence of any apparent DNA density, but is otherwise similar in overall features to the previously reported cryo-EM \cite{Grob_1281} and negative stain \cite{Liu_574} structures of human TFIID. Comparison of the coexisting canonical and rearranged structures confirms the presence of a common BC core (blue, Figure~\ref{fig:Fig3.5}A \& C). However, the position of lobe A is dramatically shifted from one side of the core to the other, where lobe A within the canonical state is within close proximity of the DNA binding site of lobe C (orange, Figure~\ref{fig:Fig3.5}A; yellow, Figure~\ref{fig:Fig3.5}C). The movement of lobe A from the canonical state to the rearranged state involves a translational component along the BC core axis, as lobe A changes its apparent connectivity from lobe C (canonical) to lobe B (rearranged). These two reconstructions from the TFIID-TFIIA-SCP sample suggest that TFIID exhibits high affinity DNA interactions only when it adopts the rearranged conformation, considering the lack of any apparent DNA density for the canonical conformation.\\
\indent The assignment of DNA as the narrow, linear density present only in the rearranged state of the TFIID-TFIIA-SCP sample was further supported by comparison of the rearranged cryo-EM structure from TFIID alone. Using the same multi-model approach described above, we obtained 3D cryo-EM reconstructions of the two alternative states of TFIID (Figure~\ref{fig:Fig3.5}B \& D). Comparison of the rearranged conformation (~50\% of selected particles) of TFIID with the rearranged structure of TFIID-TFIIA-SCP shows that there is strong density, which we ascribe to DNA, that is uniquely in the ternary complex and extends across the TFIID channel from lobe C to lobe A (Figure~\ref{fig:Fig3.5}C \& D). The structures of the canonical conformation of TFIID observed with TFIID alone and with the TFIID-TFIIA-SCP sample appear to be similar (Figure~\ref{fig:Fig3.5}A \& B), which suggests that the particles in the canonical conformation in the TFIID-TFIIA-SCP sample lack stably positioned SCP DNA.\\

\section{Validation of rearranged conformation using the free-hand test}

As discussed in Chapter 1, the published literature of TFIID structures has provided a number of different 3D models for TFIID alone \cite{Elmlund_691,Leurent_1797,Leurent_1554,Papai_539} and the structure that TFIID adopts on promoter DNA \cite{Elmlund_691,Papai_418} (Figure~\ref{fig:Compare}). Therefore, in order to ensure the validity of the 3D models presented here, the free-hand test was performed on cryo-EM data of TFIID-TFIIA-SCP. Originally proposed by Rosenthal \& Henderson \cite{Rosenthal_2003}, the free-hand test simultaneously provides an un-biased assessment of 3D map handedness and quality of particle alignments (Figure~\ref{fig:Fig3.6}) \cite{Baker_2012,Henderson_2011,Lau_2010,Rosenthal_2003}, providing an external validation of 3D maps similar to the R$_{free}$ value in x-ray crystallography \cite{Brunger_1992}. \\
\begin{figure}
\centering
\includegraphics[width=.8\textwidth]{../Ch3_figs/Fig3.6.eps}
\caption[Unbiased assessment of model handedness and particle alignments using the free-hand test]{Unbiased assessment of model handedness and particle alignments using the free-hand test. (1) Tilt image pairs are collected in the microscope at a small tilt angle, typically between 10 - 30\textdegree. (2) Necessary input files for the free hand test:  3D model (handedness to be tested), untilted stack, and tilted particle stack. (3) Align model to an untilted particle.  In the example, the Euler angle assigned to this particle is (0, 20, 0). (4) According to this starting angle (e.g. (0, 20, 0)), Euler angles are searched systematically within a given angular range.  (5) Graphical depiction of resulting cross-correlation plot from systematic Euler angle search.  The middle of the plot is the Euler angle for the untilted particle, (0, 20, 0).  Each Euler angle combination is tested and the cross-correlation is scored and recorded. Note the intense peak centered at (0, 50, 0). Accurate alignment of the untilted particle results in a angular difference between the untilted and tilted particles to be the same as that applied within the microscope (e.g. +30\textdegree). (6) This analysis is then extended for all particle tilt pairs and the peak value is recorded. Accurate alignment of the particles results in a cluster of peaks around +30°, the angle applied in the microscope for this example. Note that a model with opposite handedness would have a peak at -30\textdegree.}
\label{fig:Fig3.6}
\end{figure} 
 
\subsection{Test study on 26S proteasome}

Using the strategy outlined (Figure~\ref{fig:Fig3.6}) and computer software kindly provided by John Rubinstein (University of Toronto), I implemented the free-hand test on a well-behaved sample, the 26S proteasome from \emph{S. cerevisiae}. Previous research in the lab found that the 26S proteasome achieved the highest resolution to date for a non-helically symmetric molecule within the Nogales lab (6 - 10\AA\ ) \cite{Lander_2012}. After Gabriel Lander kindly collected cryo-EM tilt pairs, I manually extracted the particles (Figure~\ref{fig:Fig3.7}A) and aligned them using established laboratory 3D refinement parameters (see Materials \& Methods). The free-hand test results showed a narrow clustering of particle alignments at 20\textdegree\ (Figure~\ref{fig:Fig3.7}C). Satisfyingly, this is the same tilt angle applied to the sample holder in the EM, indicating that the model of the 26S proteasome has the correct handedness.  This is a trivial result considering that the high resolution of the 26S proteasome model allowed unambiguous docking of crystal structure components, confirming the handedness \cite{Lander_2012}. However, this sample still served as an important positive control for this new test, where the 26S proteasome data provided insight into the accuracy of particle alignments needed to achieve sub-nanometer resolution of single particles in cryo-EM. \\
\begin{figure}
\centering
\includegraphics[width=.67\textwidth]{../Ch3_figs/Fig3.7.eps}
\caption[Free-hand test on 26S yeast proteasome]{Free-hand test on 26S yeast proteasome. (A) Selected particle tilt-pairs.  (B) Refined 3D cryo-EM model for 26S proteasome used for free-hand test.  (C) Free-hand plot for 86 particle tilt pairs.}
\label{fig:Fig3.7}
\end{figure}
\indent 

\subsection{Free-hand test of the rearranged \& canonical conformations}
\begin{figure}
\centering
\includegraphics[width=0.9\textwidth]{../Ch3_figs/Fig3.8.eps}
\caption[Free-hand test on TFIID-TFIIA-TFIIB-SCP]{Free-hand test on TFIID-TFIIA-TFIIB-SCP. (A) Particle tilt pairs were collected at -15\textdegree and +15\textdegree on a sample of TFIID-TFIIA-TFIIB-SCP (see Chapter 4 for more information). Tilt pair plot for particles belonging to the rearranged conformation (B) and canonical conformation (C). }
\label{fig:Fig3.8}
\end{figure}
After validating the implementation of the free-hand test on the 26S proteasome, the free-hand test was performed on the rearranged and canonical cryo-EM models of TFIID-TFIIA-SCP. Particle tilt pairs (Figure~\ref{fig:Fig3.8}A) were manually selected, extracted, and subjected to identical alignment parameters that were originally used for the 3D refinement and reconstruction of the TFIID-TFIIA-SCP data (see Materials \& Methods). The free-hand test results show that the 3D model for the rearranged conformation has the corrected handedness (Figure~\ref{fig:Fig3.8}B). Furthermore, these data show that, overall, the particle alignment accuracy is correct, but, when compared to the free-hand test for the 26S proteasome, the more diffuse clustering of the rearranged particles provide insight into the limited resolution of the rearranged 3D model ( 32\AA\ ) when compared to the 26S proteasome (6 - 10\AA\ ) (Figure~\ref{fig:Fig3.8}B).\\ 
\indent The free-hand test for the corresponding canonical conformation revealed that the particle alignments were almost 'random' when compared with the rearranged model particle alignments (Figure~\ref{fig:Fig3.8}C). This result is in contrast to the previously observed agreement between projections of the canonical conformation and 2D reference-free class averages (Figure~\ref{fig:Fig3.5}E). Considering that there is an interdependence of model quality, particle quality, and 3D alignment routine that is measured by the free-hand test \cite{Baker_2012,Henderson_2011,Lau_2010}, it is difficult to understand the poor particle alignments for the canonical conformation. However, considering that the predominant conformation of within the TFIID-TFIIA-SCP sample is the newly identified rearranged conformation (Figure~\ref{fig:Fig2.6}D), the free-hand test results for the rearranged conformation provide an objective validation of the model handedness and particle alignment accuracy. This approach for validating 3D models can be extended to the other TFIID structures  \cite{Elmlund_691,Leurent_1797,Leurent_1554,Papai_539,Papai_418} so that the field can arrive at a consensus on TFIID's structure.  
  
\section{Nanogold as a tool to localize ligands bound to TFIID}

The free-hand test verified the accuracy of the 3D rearranged model from TFIID-TFIIA-SCP, allowing further experiments to be performed in order to define the stereochemistry of this ternary complex. In order to visualize the DNA within the rearranged state for labeling, all structural experiments were performed on plunge frozen samples of TFIID-TFIIA-SCP. Due to the low SNR of cryo-EM images, traditional protein-based tags (e.g. antibody, streptavidin, or MBP) may be difficult to localize accurately. Therefore, in order to overcome this problem, a Nanogold-based tagging strategy was developed to label the SCP DNA and TFIIA within TFIID-TFIIA-SCP.
 
\subsection{Theoretical considerations for imaging Nanogold within vitreous ice}

\begin{figure}
\centering
\includegraphics[width=0.65\textwidth]{../Ch3_figs/Fig3.9.eps}
\caption[Determination of optimal defocus for 1.4-nm Nanogold]{Determination of optimal defocus for 1.4-nm Nanogold. 2D Gaussians were fit to the simulated images of a 70 atom gold nanocluster embedded in water (A). For the peak finding shown here, noise was added to the simulated images to model accurately cryo-EM image characteristics (B). (C) Intensity of a 70-atom gold nanocluster (blue) is plotted against a range of defocal values. The standard deviation of the gold nanocluster intensity is shown in grey. For comparison, the intensity of protein (PDB 1PPI) is shown (red). (D) Distance between peaks found in simulated gold images and the actual position of the gold nanocluster. Standard deviation of localization is shown in grey.}
\label{fig:Gold}
\end{figure} 

Gold nanoclusters have been an important tool for EM image analysis for the past 20 years \cite{Hainfeld_2000}. From immunolocalization in cell sectioning experiments to labeled-ligands within single particle cryo-EM, the high contrast of gold relative to protein and nucleic acids makes it an ideal labeling reagent. There appears to be a wide-spread acknowledgement that gold nanoclusters display increased contrast at low defocus values compared to the high defocal values needed to image proteins in vitreous ice. However, while anecdotal evidence indicates that 0.5 - 1.0 $\mu$m is sufficient for imaging gold nanoclusters, there has not be a rigorous theoretical and experimental validation of 'best use' practices with single particles labeled with Nanogold. Therefore, we performed image simulations of gold nanoclusters to investigate the effect of different microscope parameters on Nanogold image acquisition and analysis.\\
\indent To provide a theoretical assessment of the intensity of Nanogold as a functional defocus within the EM, we performed multislice image simulations of gold nanoclusters embedded in water \cite{Goodman,Hall,Kirkland}. This approach takes advantage of the multislice approximation of image formation in the EM \cite{Goodman}, where the transmission of the incident wave is propagated through a series of thin specimen 'slices' using Fresnel propagation. Multislice imaging of biological macromolecules has been used successfully on previous projects within the lab \cite{Hall}, allowing existing computer programs to be manipulated in order to study gold nanoclusters. \\
\begin{figure}
\centering
\includegraphics[width=0.85\textwidth]{../Ch3_figs/Fig3.10.eps}
\caption[Fresnel fringe-induced amplification of gold nanocluster intensity]{Fresnel fringe-induced amplification of gold nanocluster intensity. (A) Cross-sections through simulated gold images from Figure~\ref{fig:Gold}A at 0, 0.15, 0.45 and 1.5 $\mu$m defocus. (B) Line traces from CTF-applied to a top hat function with a 9 pixel diameter. (C) Line traces from CTF-applied to a step function ('edge') centered on the edge of the gold nanocluster. Plots are shown alongside 0.45 $\mu$m gold line trace from (A).}
\label{fig:Traces}
\end{figure}
 
-Fringe interference is what gives us constrast of nanogold

-CTF amplification of the nanogold intensity at a specific defocus that is NOT the optimal defocus for a given particle diameter.

-gold is not a weak phase object--> much higher amplitude changes in electrons, higher scattering angles. this is shown by the results from a gigantic aperture = you lose the contrast of the gold!

-Gold nanocluster is like a top hat function, where the ctf amplifies the intensity of the gold. this shows that the small gold cluster does not get amplification due to its small size, and without this amplification, we don't see the undecagold.

--> to test this, remove object apeture: is the top hat like function a result of highly scattered electrons? yes: without aperture there is no signal from the gold!

we can reproduce these same results with a 'top hat' like function: simulated circle (0) amongst 1's of background
--> we see exact some pattern of intensity amplification
-when fringes of defocus are ~= to size of high amplitude scattering object with objective aperture inserted, there will be a constructive interference to amplify the intensity of the gold

\indent 

\subsection{Covalent and non-covalent labeling of SCP DNA}
\begin{figure}
\centering
\includegraphics[width=.6\textwidth]{../Ch3_figs/Fig3.12.eps}
\caption[Covalent labeling of DNA using 1.4-nm maleiomide-Nanogold]{Covalent labeling of DNA using 1.4-nm maleiomide-Nanogold. (A) Strategy for labeling promoter DNA with Nanogold. Terminal thiol moieties are incorporated into DNA to provide specific gold labeling location. (B) Chromatogram of DNA-Nanogold monitored at 260, 280, and 420 nm. (C) 2\% agarose gel of DNA-Nanogold labeling reaction.}
\label{fig:Fig3.12}
\end{figure}
In order to label DNA with Nanogold, two strategies were developed that involved either covalent or non-covalent linkages of the Nanogold to the DNA sample. Covalent labeling of the DNA involved the incorporation of a thiol moiety at the 5' or 3'-termini of the promoter DNA (Figure~\ref{fig:Fig3.12}A). After incubation of 1.4-nm monomaleiomide-Nanogold with thiol-containing SCP DNA, the sample was purified through size exclusion chromatography (Figure~\ref{fig:Fig3.12}B \& C). Strong absorbance of the sample at 420 nm indicated that the DNA was labeled with Nanogold, where the labeling efficiency was estimated to be 70\%. The covalent conjugation of Nanogold to the SCP DNA caused a shift in molecular weight shift when visualized using gel electrophoresis (Figure~\ref{fig:Fig3.12}C). This higher molecular weight complex was coincident with the strong aborbance at 420 nm on the chromatogram and was used for cryo-EM sample preparation.\\
\indent Compared to the covalent labeling protocol for DNA, non-covalent approaches proved much more challenging due to aggregation of the Nanogold reagent. Biotin-labeled DNA was utilized for these experiments in an attempt to assemble a streptavidin(SA)-Nanogold-DNA ternary complex(Figure~\ref{fig:Fig3.13}A). Unlike maleiomide-Nanogold, which was shipped in lyophilized form, SA-Nanogold was shipped resuspended in solution since the SA-Nanogold complex could not be frozen.  When freshly ordered SA-gold was incubated with biotinylated-DNA, a clean band shift was observed indicating that the streptavidin bound in a stoichiometric manner (Figure~\ref{fig:Fig3.13}B, lanes 1 \& 2).  This assembled complex proved to be very labile because, after 72 hrs, the complex was degraded where both the labeled DNA and unlabeled DNA species precipitated over time (Figure~\ref{fig:Fig3.13}B, lanes 2 \& 3). Furthermore, when this reaction was performed 1 month after ordering SA-Nanogold, higher order DNA-SA-gold complexes were visualized on the gel (Figure~\ref{fig:Fig3.13}B, lane 4). This suggested that the SA-Nanogold was precipitating over time because there was less 'freely available' SA-Nanogold for binding to the DNA. This resulted in the formation of higher ordered DNA-SA-Nanogold complexes because there was no longer a 10-fold excess of SA-Nanogold to the DNA. These data indicate that non-covalent strategies for labeling DNA were less robust than the covalent methods and, accordingly, all Nanogold-DNA labeling experiments were performed with covalent Nanogold-DNA complexes.\\
\begin{figure}
\centering
\includegraphics[width=.45\textwidth]{../Ch3_figs/Fig3.13.eps}
\caption[Attempted non-covalent labeling of DNA using streptavidin-Nanogold]{Attempted non-covalent labeling of DNA using streptavidin-Nanogold. (A) Strategy for labeling promoter DNA with streptavidin-Nanogold. Terminal biotin moieties are incorporated into DNA to provide specific gold labeling location. (B) 2\% agarose gels of DNA-SA-Nanogold complexes. }
\label{fig:Fig3.13}
\end{figure}

\subsection{Non-covalent labeling of TFIIA}
\begin{figure}
\centering
\includegraphics[width=0.45\textwidth]{../Ch3_figs/Fig3.14.eps}
\caption[Non-covalent labeling of TFIIA using Ni$^{2+}$-Nanogold]{Non-covalent labeling of TFIIA using Ni$^{2+}$-Nanogold. (A) Strategy for labeling TFIIA with Ni$^{2+}$-Nanogold. TFIIA contains an engineered His(5x) tag in addition to a naturally occurring polyhistidine sequence. Chromatogram of TFIIA-Ni$^{2+}$-Nanogold purification monitored at 280 \& 260 nm (B) and 420 nm (C). (D) SDS-PAGE gel analysis of fractions from gel filtration column. The presence of Ni$^{2+}$-Nanogold shifts the TFIIA peak dramatically into two separate peaks. The asterisk denotes the fraction used for cryo-EM grid preparation with TFIID and DNA. For clarity, only one of the two subunits of TFIIA is shown on the gel that corresponds to the fusion subunit $\alpha$-$\beta$.}
\label{fig:Fig3.14}
\end{figure}
\indent The presence of multiple cysteine residues within TFIIA, one of which that makes contacts with TBP \cite{Bleichenbacher_2003}, excluded the possibility of pursuing a covalent labeling approach with maleiomide-Nanogold.  Therefore, in order to label TFIIA with Nanogold, a non-covalent strategy was implemented that utilized the 5x-histidine tag on the gamma subunit of TFIIA to bind 1.8nm-Ni$^{2+}$-Nanogold (Figure~\ref{fig:Fig3.14}A). After screening a variety of conditions, the appropriate buffer and reaction time was determined to yield a labeled TFIIA-Nanogold complex. When comparing the migration of TFIIA alone vs. TFIIA-Nanogold using size exclusion chromatography, there is a distinct shift in the size exclusion profile to a larger molecular weight (Figure~\ref{fig:Fig3.14}B \& D). This new peak is accompanied by the presence of strong absorbance at 420 nm (Figure~\ref{fig:Fig3.14}C), thus indicating that Nanogold is present within this fraction. While non-covalent labeling TFIIA with 1.8nm-Ni$^{2+}$-Nanogold proved to be more successful than SA-Nanogold labeling of DNA, this approach still required much more optimization than the covalent DNA labeling with Nanogold, providing additional evidence that covalent labeling strategies should pursued if possible.  \\

\subsection{Strategy for cryo-EM data collection \& analysis}
\begin{figure}
\centering
\includegraphics[width=1\textwidth]{../Ch3_figs/Fig3.15.eps}
\caption[Data processing strategy for single particles labeled with Nanogold in vitreous ice]{Data processing strategy for single particles labeled with Nanogold in vitreous ice. (A) Micrographs of TFIID-TFIIA-SCP(gold) at high defocus (left) and low defocus (right). Scale bar is 200 nm. (B) Individual particles from high defocus and low defocus micrographs. 2D reference-free alignments were performed on filtered and dusted high defocus particles (top row).  These alignments will be transferred to the low defocus particles. Consistent with image simulations of gold nano-clusters, Nanogold intensity is greater in the low defocus particle than the corresponding high defocus particle.  This is indicated by red circles with peak intensity value.}
\label{fig:Fig3.15}
\end{figure}
As indicated by the image simulations of gold nano-clusters (CITE FIGURES), cryo-EM image processing of single particles labeled with Nanogold requires collecting defocal image pairs at high and low defocus (Figure~\ref{fig:Fig3.15}A). This strategy will utilize the high defocus particles to guide the 2D reference-free alignment and classification, where the resulting alignment parameters will be transferred to the low defocus particles to investigate the localization of Nanogold. In order to extract the maximum amount of information from the particle defocal pairs, the individual particle images were manipulated to highlight the location of Nanogold within the low defocus particle images. Both theoretical and experimental measurements of Nanogold in low defocus images revealed that the Nanogold intensity was routinely $>$ 4$\sigma$ above the average image intensity.. Therefore, to highlight the gold location within images, high defocus and low defocus particles were thresholded at $\sigma$ = 4, where pixels $>$ 4$\sigma$ are set to an intensity of 1 while pixels $<$ 4$\sigma$ are set to an intensity of 0 (Figure~\ref{fig:Fig3.15}B). These resulting particle images will be averaged in an identical manner to the high defocus particles in order to reveal the location of Nanogold within labeled TFIID-TFIIA-SCP complexes.   \\

\section{Defining the stereochemistry of the rearranged TFIID-TFIIA-SCP complex}

\subsection{Orientation of DNA within TFIID-TFIIA-SCP}
Before utilizing the Nanogold-labeled SCP DNA constructs for localization, we first examined the orientation of the DNA bound to the rearranged TFIID-TFIIA-SCP complex through a DNA extension experiment. Cryo-EM data were collected from a sample of TFIID, TFIIA, and SCP DNA with a 30 bp 5' extension to position -66 (termed '-66') upstream of the TATA box (Figure~\ref{fig:Fig3.17}). Comparison of cryo-EM 3D reconstructions for the rearranged state of TFIID-TFIIA-SCP(-66) and TFIID-TFIIA-SCP revealed significant extra density extending out of lobe A, which was the strongest difference between the two cryo-EM reconstructions at $\sigma$ = 4 (Figure~\ref{fig:Fig3.17}, arrowheads). The dimensions of this additional density are consistent with the length of the DNA extension. Beyond identifying the upstream DNA sequence at position -66, this additional DNA density extending from lobe A also provided a marker for the position of the TATA box, which is 36 bp from the end of this extended DNA, and thus within lobe A of the rearranged state.\\
\begin{figure}
\centering
\includegraphics[width=0.85\textwidth]{../Ch3_figs/Fig3.17.eps}
\caption[Extending SCP DNA upstream by 30 bps reveals upstream DNA path exiting lobe A]{Extending SCP DNA upstream by 30 bps reveals upstream DNA path exiting lobe A. Cryo-EM reconstruction of TFIID-TFIIA-SCP(-66) at 35 \AA\ (A) aligned with the cryo-EM reconstruction of TFIID-TFIIA-SCP (B) and their difference map (C). DNA density is colored in green, whereas lobe A and the BC core are colored in yellow and blue, respectively. Summary of Nanogold labeling from Figure~\ref{fig:Fig3.16} is shown in (A). Scale bar is 100\AA\ . }
\label{fig:Fig3.17}
\end{figure}
\indent We further examined the orientation of promoter DNA on TFIID by covalent labeling of the SCP DNA with 1.4-nm maleimido-Nanogold at either +45 or -36 (TATA box) on the SCP promoter sequence. This strategy takes advantage of the high contrast of Nanogold relative to that of protein and DNA at low defocus, providing a strong signal for the localization of gold particles relative to protein and DNA density. Analysis of the 2D reference-free class averages for high defocus particles of TFIID-TFIIA-SCP(+45 gold) and subsequently applying the alignments to the low defocus particles revealed that the downstream region of the SCP is bound by lobe C (Figure~\ref{fig:Fig3.16}A). This is indicated by the narrow clustering signal from the low defocus thresholded particles (Figure~\ref{fig:Fig3.16}A, right). When this localization is mapped onto the 3D structure of the rearranged conformation, the localization of +45 gold is consistent with the DNA extension experiment because the TFIID-TFIIA-SCP(+45 gold) average labels the DNA on a different face of the TFIID structure (Figure~\ref{fig:Fig3.17}A). Moreover, the length of DNA between the +45 gold label and lobe C (Figure~\ref{fig:Fig3.17}A) is 15 bps, the same distance between the DPE and +45 site on the SCP.\\ 
\indent To provide an orthogonal labeling strategy for the TATA box, cryo-EM data were collected on a sample of TFIID-TFIIA-SCP(TATA gold) where the TATA box was covalently labeled with Nanogold (Figure~\ref{fig:Fig3.16}B). When the high defocus alignments were applied to the low defocus thresholded particles, there was a cluster of signal within lobe A at a location consistent with DNA extension experiments. Thus, these SCP DNA labeling experiments, which are summarized in Figure~\ref{fig:Fig3.17}A, have defined the organization of TFIID bound to promoter DNA, providing the first structural insight into promoter binding by human TFIID. \\
\begin{figure}
\centering
\includegraphics[width=0.9\textwidth]{../Ch3_figs/Fig3.16.eps}
\caption[Organization of promoter DNA and TFIIA within the TFIID-TFIIA-SCP complex]{Organization of promoter DNA and TFIIA within the TFIID-TFIIA-SCP complex. 2D reference-free class averages for the rearranged TFIID-TFIIA-SCP complex with Nanogold labels on SCP DNA at +45 (A), SCP DNA at TATA (B), and TFIIA (C). (D) 2D reference-free class average for the canonical state of TFIID-TFIIA-SCP containing Nanogold labeled on TFIIA. 3D models are shown alongside high defocus averages (density threshold at $\sigma$ = 3.5) with a gold sphere marking the localization of Nanogold for each experiment. }
\label{fig:Fig3.16}
\end{figure}

\subsection{TFIIA localizes to lobe A in both the canonical and rearranged conformations of TFIID}
We next investigated the location of TFIIA within the rearranged TFIID-TFIIA-SCP complex by analysis of TFIIA that was labeled with 1.8-nm Ni$^{2+}$-NTA-Nanogold. Upon 2D reference-free analysis of the high defocus cryo-EM particles, the determined alignment parameters were applied to the low defocus focal mate and class averages were calculated. Nanogold densities identified the TFIIA binding site within lobe A of the rearranged state (Figure~\ref{fig:Fig3.16}C), adjacent to the TATA Nanogold location (Figure~\ref{fig:Fig3.16}B). Importantly, analysis of 2D reference-free class averages corresponding to the canonical state also demonstrated that TFIIA localized to lobe A in its alternate configuration relative to the BC core (Figure~\ref{fig:Fig3.16}D). These findings are in agreement with the well-characterized direct interaction of TBP and TFIIA, and further demonstrate, in the context of TFIID, that TFIIA interacts with the complex at a location close to the TATA box-binding site. These findings therefore suggest that both TBP and TFIIA are localized within lobe A during the structural transition between canonical and rearranged states of TFIID.

\subsection{Model of promoter DNA path through TFIID-TFIIA-SCP}

The results presented within this chapter have structurally characterized and validated the three-dimensional structure of TFIID bound to TFIIA and promoter DNA within the rearranged conformation. This process involved calculating an \emph{ab initio} 3D reconstruction of the rearranged conformation (Figure~\ref{fig:Fig3.3}) that was simultaneously refined in a multi-model fashion against cryo-EM data to provide both canonical and rearranged reconstructions of TFIID-TFIIA-SCP (Figure~\ref{fig:Fig3.5}). Given the dramatic nature of the structural rearrangement, we wanted to provide an objective measure of the accuracy of the rearranged 3D model bound to promoter DNA. Therefore, we implemented the free-hand test to assess the model handedness and quality of particle alignments. This test confirmed that the handedness of the rearranged model was correct, in addition to revealing that the model was able to align $>$ 80\% of the tilted particles into the overall correct orientation (Figure~\ref{fig:Fig3.8}B).  \\
\indent After confirming the accuracy of the rearranged model of TFIID-TFIIA-SCP, labeling experiments were used to define the orientation of DNA and TFIIA on TFIID. By integrating these labeling data within the structure of the DNA-bound rearranged state, we have developed a model to describe the DNA path through the modular sub-domains of the TFIID-TFIIA-SCP complex (Figure~\ref{fig:Fig3.18}). Gold labeling of the SCP DNA at the +45 position indicated that the density extending off of lobe C is the downstream promoter DNA (Figure~\ref{fig:Fig3.16}A). It thus appears that the MTE and DPE promoter elements are bound by lobe C (Figure~\ref{fig:Fig3.18}A), which suggests that lobe C contains subunits TAF6 and TAF9 \cite{Burke_2739, Theisen_341}. Further support for this conclusion came through antibody localization of $\alpha$-TAF6 antibody to lobe C (Appendix B, Figure~\ref{fig:TAF6}). Given the linear nature of the DNA contacts with lobe C, we can conclude that the histone-fold containing subunits TAF6 and TAF9 do not interact with DNA in a nucleosome-like manner. This is consistent with recent biochemical studies that reconstituted a TAF6/9/4/12 tetrameric complex that was capable of interacting with DPE-containing promoter DNA. Interestingly, DNA-binding was only disrupted when loops extending away from the histone fold domains of TAF6 and TAF9 were truncated, but not when mutations were introduced within the core histone fold domains  \cite{Shao_1340}. Thus, the histone-fold containing subunits TAF6 and TAF9 are located within lobe C and interact with promoter DNA in a linear manner that is inconsistent with nucleosome-like distortions of DNA. \\ 
\indent Across the central channel from lobe C, lobe A interacts with SCP DNA extending from the Inr to the TATA box (Figure~\ref{fig:Fig3.18}A). Previous work demonstrated that this region of the core promoter interacts with a TAF1, TAF2, and TBP sub-complex that is also able to direct transcription from TATA-Inr promoters \cite{Chalkley_2339}. Given the modularity of lobe A within the context of TFIID, it is likely that lobe A contains TAF1, TAF2, and TBP and serves as a modular domain of TFIID that is capable of interactions with promoter DNA upstream of the TSS. \\
\indent This proposed composition of lobe A is also consistent with studies addressing the integrity of the TFIID complex \emph{in vivo} \cite{Wright_1170}.  Through systematic RNAi-mediated knock-down of TBP and TAFs within Drosophila S2 cells, the authors defined a stable sub-complex of TFIID that was nucleated by TAF4.  Moreover, RNAi knock-down of TAF1, TAF2, or TBP did not affect the integrity of the TFIID complex, suggesting that these subunits were located on the periphery of TFIID.  We believe that these data provide independent support for the conclusion that lobe A comprises TAF1, TAF2, and TBP, existing as a modular domain of TFIID.  Furthermore, this study identified the composition of the stable TFIID sub-complex to be TAF4, -5, -6, -9 and -12.  We propose that this stable sub-complex corresponds to the BC core identified in this study, providing a framework for further studies addressing the contributions of the BC core to TFIID function.  \\
\begin{figure}
\centering
\includegraphics[width=.8\textwidth]{../Ch3_figs/Fig3.18.eps}
\caption[Structural description of the rearranged TFIID-TFIIA-SCP complex relative to the TSS]{Structural description of the rearranged TFIID-TFIIA-SCP complex relative to the TSS. (A) Promoter DNA for SCP(-66) docked into the TFIID-TFIIA-SCP(-66) map (shown in mesh). DNA model corresponds to sequences from -66 to +45. (B) The proposed location of the crystal structures of TBP-TFIIA-TFIIB on TATA box DNA is in close proximity to Inr (PDB accession codes 1VOL and 1NVP). The unresolved DNA path between Inr and TATA is indicated by a dotted line.}
\label{fig:Fig3.18}
\end{figure}
\indent To model the DNA path through lobe A, two bends were incorporated to accommodate the 120\textdegree\ angle that TFIID imposes on the downstream and upstream DNA regions (Figure~\ref{fig:Fig3.18}A). Initial attempts to trace the DNA path through the TFIID-TFIIA-SCP structure using a single bend at the TATA box failed to generate a model that was compatible with the experimentally obtained positions of downstream and upstream DNA locations (data not shown). Given the angle of upstream DNA and the location of the -66 position, we modeled the location of TBP and TATA box DNA to be 36 bp (12.2 nm) downstream from the -66 position observed in the TFIID-TFIIA-SCP(-66) structure. After incorporating this bend at -31/30, a second gradual bend is proposed to form along the DNA between the TATA box and Inr. We suggest that this DNA deformation between the TATA box and Inr could play a role in positioning TBP and TFIIB in close proximity to the TSS for efficient loading of RNAPII. \\
\indent Given that the above structural analyses did not provide a direct probe for the location of the TATA box within lobe A, we collected more data cryo-EM data for TFIID-TFIIA-SCP(TATA gold) to try to localize extra density corresponding to the Nanogold within the high defocus particles. This study was motivated by the strong additional density that appeared at the +45 location on the SCP within the high defocus class average for TFIID-TFIIA-SCP(+45 gold) (Figure~\ref{fig:Fig3.16}A). After collecting and analyzing a large dataset for TFIID-TFIIA-SCP(TATA gold), a strong additional density appeared on lobe A within the high defocus averages of rearranged state bound (Figure~\ref{fig:Fig3.19}A \& B, left). When 2D difference maps were calculated for between these reference-free class averages and TFIID-TFIIA-SCP class averages that did not contain Nanogold, a strong difference peak ($>$ 4$\sigma$) was observed. The strength and size of this difference peak indicates that the additional density seen within class averages of TFIID-TFIIA-SCP(TATA gold) corresponds to Nanogold. This finding is in close agreement with the cluster of peaks seen in the thresholded low defocus particles (Figure~\ref{fig:Fig3.16}B).\\ 
\begin{figure}
\centering
\includegraphics[width=.8\textwidth]{../Ch3_figs/Fig3.19.eps}
\caption[Localization of TATA-Nanogold within high defocus class averages]{Localization of TATA-Nanogold within high defocus class averages. (A) \& (B) 2D reference-free class averages for high defocus, dusted particles from TFIID-TFIIA-SCP(TATA gold) aligned with 2D difference map and model projection. The 2D difference map was obtained upon comparison of these averages with TFIID-TFIIA-SCP averages, where the difference density exhibits $>$ 4$\sigma$ intensity value, indicating significance. Considering that these two averages were obtained from different Euler angles, back-projection of the difference density reveals location of TATA gold within lobe A (C - E). (F) A 70 atom gold nano-cluster \cite{Jadzinsky_2007} is docked as a reference for 1.4 nm-Nanogold size and location. Note that the location of Nanogold is approximately 20 \AA\ away from the intersection of the back-projected difference densities.}
\label{fig:Fig3.19}
\end{figure}
\indent Since two different orientations were found to contain this additional density, alignment of projections from the rearranged 3D model from TFIID-TFIIA-SCP allowed back-projection of the difference density (Figure~\ref{fig:Fig3.19}C - E). Superimposing the back-projection of the two related, but distinct, orientations of the difference densities provided a precise location of the TATA box within the rearranged TFIID-TFIIA-SCP(-66) structure. It is important to consider that the Nanogold was covalently linked to the SCP sequence at -36, placing the Nanogold 15\AA\ upstream of the TATA box location (-31/30).  Furthermore, the Nanogold was tethered to the DNA through a six carbon linker (9\AA). Thus, there is approximately 25\AA\ between the TATA box and the Nanogold label. Modeling the size and location of Nanogold within the previously proposed model of TFIID-TFIIA-SCP(-66) shows that the modeled location of the TATA box is within 20\AA\ of the experimentally determined TATA box location (Figure~\ref{fig:Fig3.19}F). Given this new localization data, we should be able to provide a more accurate model of the DNA path through lobe A. However, it is difficult to utilize this new labeling information while also using the characteristic kink of DNA within TBP-TATA \cite{Kim_3416,Kim_3377} and the extension of upstream DNA exiting lobe A seen within the structure of TFIID-TFIIA-SCP(-66). \\
\indent Precise modeling of the promoter DNA path and topology through TFIID will require a determining a structure of TFIID-TFIIA-SCP at a higher resolution (15\AA) in order to visualize the major and minor grooves of the promoter DNA. The structures presented here were unable to surpass 32\AA\ resolution (Figure~\ref{fig:Fig3.5}). Since these structures were calculated from a dataset containing 34,167 particles, we hypothesized that collecting a much larger dataset should increase the resolution of the structures. After collecting and analyzing a dataset that contained a combined total of 150,000 particles, 3D model refinements did not result in any detectable resolution improvement (data not shown). This suggests that high resolution structural studies of TFIID-TFIIA-SCP will require improving particle contrast within the cryo-EM images, since the presence of glycerol, sugars, and a carbon support likely limit the SNR of the individual particles. Recent advances in direct electron detection technology \cite{Campell_2012,Milazzo_2011} in combination with phase-plates \cite{Nagayama_2011} hold the promise of increasing the SNR for cryo-EM images and will likely be necessary tools to reach higher resolution.\\


