\section{Image analysis}

Particles from the negative stain data shown in Figure 2 were manually picked using Boxer (EMAN) (Ludtke et al., 1999), CTF-estimated using CTFFIND3 (Mindell and Grigorieff, 2003), and phase-flipped using SPIDER (Frank et al., 1996). Cryo-EM data were prepared using the Appion image processing environment (Lander et al., 2009) where particles were automatically picked using Signature (Chen and Grigorieff, 2007), CTF-estimated using CTFFIND3 (Mindell and Grigorieff, 2003), phase flipped using SPIDER (Frank et al., 1996), and normalized using XMIPP (Sorzano et al., 2004). 
	For negative stain, 2D reference-free image analysis was performed using an iterative routine implementing a topology representing network (Ogura et al., 2003) in combination with mutli-reference-alignment within IMAGIC (van Heel et al., 1996). For cryo-EM, 2D reference-free image analysis was performed exclusively within IMAGIC (van Heel et al., 1996) through iterative rounds of multivariate statistical analysis and multi-reference-alignment. 
	3D refinements were performed on phase-flipped particles using an iterative projection matching and 3D reconstruction approach that was performed using libraries from EMAN2 and SPARX software packages (Hohn et al., 2007; Tang et al., 2007). 34,167 particles from cryo-EM grids prepared from TFIID-TFIIA-SCP sample were refined against low-pass filtered models of the canonical and rearranged state. For TFIID, 30,800 particles were collected and refined. Following this global projection matching routine, all data were further refined in FREALIGN (Grigorieff, 2007) and low-pass filtered at the final resolution while applying a negative B-factor using bfactor.exe. 


\subsection{Orthogonal tilt reconstruction of TFIID-IIA-SCP} 
The substantial structural differences between the canonical and rearranged states of TFIID rendered the previously reported 3D reconstruction of the canonical state human TFIID unable to be used as a reference for projection matching and 3D reconstruction.  We thus generated an ab initio model for the rearranged state by implementing the orthogonal tilt reconstruction (OTR) technique (Leschziner and Nogales, 2006) on negatively stained samples of TFIID-TFIIA-SCP. \\
\indent Tilt pairs at +/- 45° were collected on a TFIID-IIA-SCP sample prepared on glow discharged holey carbon film (Bradley, 1965) on a 400 mesh copper grid (Electron Microscopy Sciences) with a thin continuous carbon layer using a Tecnai F20 TWIN transmission electron microscope operating at 120 keV using a dose of 25 e-/Å2 on a Gatan 4k x 4k CCD at a nominal magnification of 80,000X (1.501 Å/pixel).  Particles were manually picked from the tilt pairs using ‘xmipp_mark’ within XMIPP (Sorzano et al., 2004).  The XMIPP coordinates (.pos) were converted to BOXER coordinates (.box) prior to particle extraction using ‘batchboxer’ within EMAN (Ludtke et al., 1999).  The tilt angles for each micrograph were calculated using CTFTILT (Mindell and Grigorieff, 2003).  Extracted particles from a given tilt were subjected to 2D reference-free classification and alignment using multivariate-statistical-analysis and multi-reference-refinement within IMAGIC (van Heel et al., 1996).  To calculate individual class volumes, euler angles were calculated using the equations previously described (Leschziner, 2010), 3D reconstructions were performed using ‘bp 3f’ within SPIDER (Frank et al., 1996), and the volumes were filtered based upon FSC = 0.5.   
Since OTR preserves three-dimensional features of individual class volumes without suffering from the ‘missing-wedge’ problem of random conical tilt (Chandramouli et al., 2011), individual class volumes were refined against untilted negative stain data for TFIID-IIA-SCP.  These refined models were then quantitatively compared to the characteristic class averages of TFIID in six characteristic conformations (Figure S2D) and models were excluded based upon Fourier Ring Correlation (FRC) (Saxton and Baumeister, 1982) values below 60 Å (Figure S2E).  A majority (8/10) of the refined, validated models were found to be in the rearranged state (Figure S2F, c – g), whereas a minority (2/10) were in the canonical conformation (Figure S2F, a & b).  

\subsection{Focused classification of lobe A}
The overall data processing strategy for assessing lobe A’s conformational heterogeneity is mapped out in Figure S1B & C.  Briefly, negatively stained particles for each dataset – TFIID (16,000 particles), TFIID-SCP (9,859 particles), TFIID-IIA-SCP (19,678 particles), and TFIID-IIA (20,152 particles) – were aligned and classified in a reference-free fashion using a topology representing network (Ogura et al., 2003) and multi-reference-alignment within IMAGIC (van Heel et al., 1996), where class average reference images were aligned to each other and mirrors generated after each iteration using the ‘prep-mra-refs’ command within IMAGIC.  After 8 – 9 iterations, all particles corresponding to the characteristic view of TFIID were extracted, mirrored if necessary, and placed into a single stack of particles.  These particles were then aligned to a masked reference of the BC core using multi-reference-alignment within IMAGIC (van Heel et al., 1996).  Multivariate-statistical-analysis was performed within a mask drawn around lobe A.  The resulting eigen-images reflected the conformational flexibility of lobe A (Figure S1B) and were used to perform hierarchical ascendant classification.  Subsequently generated classums (10 – 20 particles/class) showed a range of positions for lobe A relative to the rigid BC core (Figure S1C).
	To measure lobe A’s position, individual classums were measured using BOXER within EMAN (Ludtke et al., 1999) and plotted within MATLAB (MathWorks, 2011).  Boxes were placed, in order, on lobes B, C, and A.  The coordinates were used to calculate a projection of lobe A’s position on the B-C axis that were normalized to the length of the BC core (Figure S1E).  A histogram of lobe A positions were calculated within MATLAB and the distribution was fitted using a maximum-likelihood estimate of a mixed Gaussian distribution (‘gmdistribution.fit’) and was plotted using ‘pdf.’  Wilcoxon rank sum tests were performed within MATLAB (‘ranksum’) to test for statistical differences between distributions of lobe A positions from the different datasets.

\subsection{Cryo-EM of Nanogold complexes \& image analysis} 
For localization of Nanogold labels within cryo-EM images of TFIID-IIA(gold)-SCP (10,075 particles), TFIID-IIA-SCP(Nanogold @ TATA) (851 particles), and TFIID-IIA-SCP(Nanogold @ +45) (6,502 particles), focal pair images were collected using Leginon (Suloway et al., 2005) at defocuses of -3 μm and -0.5μm (Figure S5C & D) on a Tecnai F20 TWIN transmission electron microscope operating at 120 keV using a dose of 25 e-/Å2 on a Gatan 4k x 4k CCD at a nominal magnification of 80,000X (1.501 Å/pixel).  Particles were manually picked using BOXER, shifts between focal pairs were calculated using ‘alignhuge’ and applied to the particle coordinate files prior to particle extraction with ‘batchboxer’ within EMAN (Ludtke et al., 1999).  The high defocus particles were then normalized and dusted using ‘xmipp_normalize’ within XMIPP (Sorzano et al., 2004) to remove gold signal > 3.5σ.  These dusted, high defocus particles were subjected to reference-free 2D classification using iterative classification and alignment within IMAGIC with multivariate statistical analysis and multi-reference-alignment (van Heel et al., 1996).  The rotation and shifts applied to the high-defocus particles were applied to the following stacks of particles using ‘equivalent-rotation’ within IMAGIC:  1) un-dusted high defocus particles termed ‘high defocus particles’ that were normalized using ‘edgenorm’ within the proc2d command of EMAN; 2) thresholded high defocus particles where pixels > 4σ were converted to 1 and all pixels below 4σ were converted to 0 for each un-dusted high defocus particle; 3) un-dusted low defocus particles termed ‘low defocus particles’ that were normalized using ‘edgenorm’ within the proc2d command of EMAN; and 4) thresholded low defocus particles where pixels > 4σ were converted to 1 and all pixels below 4σ were converted to 0 for each un-dusted low defocus particle.  By averaging low defocus particles and low defocus thresholded particles according to the 2D reference-free alignment of dusted high defocus particles, Nanogold density could be localized with high confidence given the high contrast of Nanogold at low defocus values.

\subsection{Cryo-EM of TFIID-TFIIA-SCP(mTATA)}
For structural analysis of mutant TATA box promoter, TFIID was incubated with 10X excess of SCP(mTATA) and TFIIA, under identical conditions as TFIID-TFIIA-SCP sample preparation.  27,500 particles were collected and analyzed. 
