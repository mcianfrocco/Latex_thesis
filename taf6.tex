%\chapter*{Appendix A: TAF6 antibody localization}
%\addcontentsline{toc}{chapter}{Appendix A: TAF6 antibody localization}
%\chaptermark{Appendix}
%\markboth{Appendix}{Appendix}

\chapter{Antibody labeling of TFIID}

Given the lack of information regarding the subunit organization within TFIID, there has been a concerted effort by a number of laboratory members to generate and analyze TFIID-antibody datasets. Most of these datasets were prepared and analyzed by Wei-Li Liu (Assistant Prof., Albert Einstein College of Medicine) and Patricia Grob (staff scientist). Antibodies directed against TBP, TAF4, and TAF6 were bound to TFIID during the purification procedure, allowing the elution of stoichiometric TFIID-antibody complexes. Unfortunatley, all previous attempts to localize these antibodies were unsuccessful. \\
\indent After describing the extent of lobe A's flexibility within the context of TFIID alone, we hypothesized that the flexibility of lobe A hampered previously analysis of antibody-bound TFIID particles. Therefore, TFIID-antibody datasets for $\alpha$-TBP (unpublished), $\alpha$-TAF4 \cite{Liu_723}, $\alpha$-TAF6 (unpublished) were reanalyzed after classification of lobe A into discrete conformational sub-states. While we were unable to recover any density within 2D class averages or 3D models for $\alpha$-TBP and $\alpha$-TAF4 antibodies (data not show), we were able to localize significant antibody density for $\alpha$-TAF6 bound to the BC core. This localization supports the model of DNA-bound TFIID in addition to proposing a pseudo-symmetry within the BC core that necessitates further study. 
\begin{figure}
\centering
\includegraphics[width=.7\textwidth]{../Ch4_figs/Purification.eps}
\caption[Purification strategy for TFIID-$\alpha$-TAF6]{Purification strategy for TFIID-$\alpha$-TAF6. (A) Strategy for purification of a ternary TFIID-$\alpha$-TAF6 complex. (B) SDS-PAGE of TFIID-$\alpha$-TAF6. Note that this was photo-copied from Patrica Grob's laboratory notebook. }
\label{fig:Purify}
\end{figure}

\section{Purification of a ternary TFIID-$\alpha$-TAF6 complex}

As previously stated, the purification and analysis of TFIID-$\alpha$-TAF6 was performed previously by Wei-Li Liu and Patricia Grob, respectively. The general strategy for antibody labeling of TFIID was to purify TFIID from Hela nuclear extract using established protocols, but, before elution of TFIID from the agarose beads, labeling antibodies were incubated with TFIID. After extensive washing, these TFIID-antibody complexes were eluted from the beads and analyzed using SDS-PAGE to verify antibody binding. \\
\indent Using this strategy, Wei-Li Liu purified a stoichiometric TFIID-$\alpha$-TAF6 complex with a monoclonal antibody directed against TAF6 (Figure~\ref{fig:Purify}A). Analysis of this labeling through SDS-PAGE revealed strong binding by the $\alpha$-TAF6 antibody that could be co-immunpurified with TFIID (Figure~\ref{fig:Purify}B). 

\section{$\alpha$-TAF6 antibody localization}

Cryo-EM analysis showed that lobe C interacts with the MTE and DPE, suggesting that TAF6 is found within lobe C since TAF6 cross-links to both MTE and DPE sequences \cite{Burke_2739,Lim_1522}. Therefore, we expected to find the TAF6 antibody bound to lobe C. But, considering the large degree of conformational flexiblity exhibited by TFIID alone (Figure~\ref{fig:Fig2.6}A), focused classification of lobe A was first performed before any attempt at antibody localization. To this end, the purified TFIID-$\alpha$-TAF6 complex was prepared for negative stain single particle EM by Patricia Grob. Particles were picked, extracted, and phase-flipped prior to storage of the dataset within the archives of the Nogales lab. These particles were subjected to careful 2D reference-free image analysis and processed as previously described (Figures~\ref{fig:Fig2.5}). Briefly, extracted particles corresponding to characteristic view were re-aligned to the rigid BC core, where the overall average of the re-aligned particles showed the characteristic flexibility of lobe A (Figure~\ref{fig:TAF6}A). After focused classification of lobe A and measurement of lobe A's position relative to the BC core, the histogram of lobe A positions indicated that the canonical and rearranged conformations were equally populated (Figure~\ref{fig:TAF6}B), as seen previously for TFIID alone (Figure~\ref{fig:Fig2.6}A). \\
\indent Since the position of lobe A within the canonical state is bound to lobe C in close proximity to the DNA binding site, we hypothesized that the epitope recognized by $\alpha$-TAF6 may be occluded within the canonical conformation. Therefore, particles were separated into groups corresponding to the canonical and rearranged conformations before further analysis. After systemtically moving a circular mask around lobe C and the BC core, we found strong antibody density only when the mask was positioned around lobe C (Figure~\ref{fig:TAF6}C). This density was found in two different sub-classified averages, providing evidence that TAF6 is localized to lobe C. \\
\indent During sub-classification of the canonical state, we noticed additional density that appeared bound to lobe B. Further analysis confirmed that, within the canonical state, lobe B interacts with the TAF6 antibody (Figure~\ref{fig:TAF6}D). This result is surprising considering that lobe C is the MTE/DPE interacting domain within TFIID. However, considering that a number of subunits within TFIID may exist as multiple copies, we believe that this localization may provide insight into the apparanet pseudo-symmetry of the BC core.
\indent 
\begin{figure}
\centering
\includegraphics[width=.8\textwidth]{../Ch4_figs/TAF6.eps}
\caption[$\alpha$-TAF6 localizes to lobes B and C within the BC core]{$\alpha$-TAF6 localizes to lobes B and C within the BC core. 2D reference-free analysis was performed previously described in Figure~\ref{fig:Fig2.6}. (A) Overall average of re-aligned particles within characteristic view. (B) Histogram of lobe A positions relative to the BC core. (C) Sub-classification of the rearranged conformation. Note additional density bound to lobe C. Averages were aligned to projections of the rigid BC core. (D) Sub-classification of the canonical conformation. Note additional density bound to lobe B. (E) Model of TFIID's subunits. Tentativity locations of TAF4, -5, -9, and -12 are shown as dotted lines.  }
\label{fig:TAF6}
\end{figure}

\section{Comparison of antibody labeling between yeast and human TFIID}

The data presented within this Appendix indicate that the $\alpha$-TAF6 antibody localizes to lobes B and C within the canonical and rearranged states, respectively. These $\alpha$-TAF6 antibody densities are the strongest example of antibody localization within human TFIID to date. The antibody localization was successful only after extensive 2D reference-free classification and subclassification of TFIID within distinct structural states. We believe that this strategy will be generally applicable for localization of additional antibodies and other TFIID-interacting factors.\\
\indent Previous research on yeast TFIID has proposed the existance of TAFs existing in multiple copies within TFIID \cite{Leurent_1554,Leurent_1797}. Indeed, mass spectrometry has supported this proposal, providing biochemical evidence for this model \cite{Sanders_2002}. Combining the biochemical characterization with antibody labeling of TFIID, a number of papers have proposed yeast TFIID to exist as a nearly-symmetric molecule \cite{Leurent_1554,Papai_191}. However, considering that the antibody-labeling of yeast TFIID employed the same strategies as previously employed on human BTAF1 and TFIIH, we propose that this error-prone strategy, combined with conformational flexibility, led to an over-symmetrized model of TFIID subunit organization.\\
\indent The same research group that performed antibody labeling of yeast TFIID also utilized a similar stategy on human BTAF1 \cite{Pereira_2004} and human TFIIH \cite{Schultz_2000}, where subsequent studies of these protein complexes revealed that these subunit locations were inaccurate. For instance, the BTAF1 labeling proposed that TBP localizes to the extension protruding away from the core of the molecule \cite{Pereira_2004}, however, x-ray crystallographic studies of yeast Mot1-TBP revealed that TBP is bound within the core domain \cite{Pereira_2004}. This suggests that, within the defined molecular system of BTAF1 (two subunits), the authors inaccurately localized TBP, the key component within BTAF1. \\
\indent To tackle an even more challenging system, the authors performed antibody localization for human TFIIH, a multi-subunit GTF responsible for promoter opening by RNAPII \cite{Schultz_2000}. It should be stated that this was a challenging system and it provided the first 'glimpse' into TFIIH architecture, however, all antibody labeling performed (\cite{Schultz_2000}: Figure 5) was been proven incorrect through structural analysis of an assembled pre-initiation complex containg  RNAPII bound to promoter DNA and TFIIH (Yuan He, Nogales laboratory, unpublished). For TFIIH, the authors did not realize that the kinase domain was extremeley flexibile, which made it nearly invisible for a number of different class averages shown (\cite{Schultz_2000}: Figure 5A - F). Furthermore, when the kinase domain was visible, it was assigned to antibody density (\cite{Schultz_2000}, Figure 5G \& H). As previously stated, this was a challenging sample, however, it serves as another instance of incorrect antibody localization by this research group.\\
\indent Given these considerations, we hesitate to extend the yeast TFIID antibody labeling studies into our model of human TFIID. While the authors are not intentionally misrepresenting their data for antibody labeling of yeast TFIID, we believe that their stategy has a false-positive rate of 40 - 50\% based upon their past work \cite{Pereira_2004,Schultz_2000}. We believe that this false-positive rate is the result of two error-prone methods employed by this research group: 1) failure to purify ternary antibody-protein complexes, and 2) 'single-particle' counting of antibodies. For human BTAF1 \cite{Pereira_2004}, human TFIIH \cite{Schultz_2000}, and yeast TFIID \cite{Leurent_1554,Leurent_1797}, the authors did not purify ternary complexes of antibody-protein complexes, instead mixing the antibody in 5 - 10 molar excesses with the protein prior to grid preparation. We believe that this method will provide false positives given the high molar excess of antibodies. While non-specific antibody-protein complexes should be averaged out when 2D reference-free class averages are calculated, the authors do not calculate averages of their labeled complexes. Instead, the authors perform 'single-particle' counting of antibody-protein complexes, where individual particle images are inspected for potential antibody density and, after investigating  600 - 1000 individual particles, locations are proposed.  Therefore, by adding the antibodies in excess and relying on localization within low SNR images of single particles, the authors inherently bias their analysis towards a high false-positive rate. \\
\indent Antibody labeling is a notoriously difficult technique for subunit localization. The above papers represent hontest attempts to identify subunit locations within biologically important protein complexes. However, for reasons listed above, it is difficult to assess the quality of subunit location assignments. Indeed, antibody localization is extremeley difficult for human TFIID as well. The localization presented here for TAF6 was the only success out of three different datsets. While the localization of TAF6 may not be entirely accurate, it serves as a starting point for future studies, where orthogonal approaches should be used to verify antibody labeled locations. For example, considering that lobe C binds the MTE/DPE , a region of DNA known to interact with TAF6 through biochemical studies, while also showing density for the $\alpha$-TAF6 antibody, we have a high degree of confidence that TAF6 is localized within lobe C. \\

\section{Implications of a pseudo-symmetric BC core}

\indent While we need to continue to verify the location of TAF6 within lobe B using alternative strategies, it is tempting to speculate on this pseduo-symmetry within the BC core. As previously noted, studies of yeast TFIID proposed the existance of a nearly-symmetric TFIID molecule, due to the presence of antibody density on both lobes. We believe that this symmetric model of yeast TFIID is result of mislocalization of antibodies within yeast TFIID. Lobe A within yesat TFIID may exhibit the same large degree of flexibility as observed for human TFIID, necessitating that all analysis of yeast TFIID labeled with antibodies be obtained from class averages. \\
\indent While the symmetric model of yeast TFIID may not be accurate, the labeling of TAF6 to both lobes B an d C suggests that the BC core may exist as a pseudo-symmetric backbone of TFIID. Initial structural inspection of the BC core revealed that it almost appeared as a C2 symmetrical structure, with the symmetry axis is located along the ridge connecting lobes B and C. This observation is not sufficient for concluding a pseudo-symmetric BC core, but the dual labeling of the TAF6 antibody to both lobes B and C provides biochemical evidence for such a model. This observation of a dual-antibody labeled TFIID molecules was only apparent \emph{after} calculating 2D averages of human TFIID and performed careful sub-classification, a detail that would be overlooked when investigating single particles for yeast TFIID.\\  
\indent The presence of TAF6 within both lobes B and C may provide a binding site for lobe A within the canonical and rearranged states. Since the discovery of lobe A's dramatic flexibilty, we posited that TFIID may contain similar binding sites for lobe A within both canonical and rearranged conformations. Therefore, after localizing TAF6 to both lobes B and C, we propose that TAF6 may serve as the common binding site for lobe A within the canonical (lobe C) and rearranged (lobe B) conformations. While this hypothesis remains untested, it serves as a reasonable molecular explanation for lobe A's ability to form the canonical and rearranged states.\\
\indent The pseudo-symmetry of the BC has been incorporated into a model describing the organization of human TFIID (Figure~\ref{fig:TAF6}E). TAF6 occupies a position within both lobes B and C, where it is likely bound to TAF9 in both cases through its highly conserved histone-fold domain \cite{Shao_1340}. From biochemical evidence, we believe that TAF6 and TAF9 further interact with TAF4 and TAF12 within lobe C to form a histone-like tetramer, an interaction shown to stabilize DPE-interactions within a TAF6/9/4/12 complex \cite{Shao_1340}. We propose that lobe C is connected to the BC core through an extension of TAF4 from the histone-tetramer. This is consistent with TAF4 serving a scaffolding role within \emph{Drosophila} TFIID, where RNAi knockdown of TAF4 results in the loss of remaining subunits within holo-TFIID \cite{Wright_1170}. Serving to facilitate the connections between lobes C and B, we believe that TAF5's dimerization domain may be localized within the ridge connecting lobes C and B. While this model remains to be tested, we believe that it serves as a valid initial description of TFIID's organization, providing a molecular basis for the pseudo-symmetry of the BC core and dual docking site for lobe A in both states. 
