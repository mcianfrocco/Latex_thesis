\section{DNase I and MPE-Fe footprinting}
Extended-length M13 sequencing primers (28 nt) with 5’-ends corresponding to basepair -109 and +110 (relative to the start site of SCP as +1) were purified by denaturing gel electrophoresis, 5’-end-labeled with [γ-32P]-ATP and T4 polynucleotide kinase, desalted on a 900 μl Sephadex G50 fine column, extracted with phenol/CHCl3/isoamyl alcohol (IAA), precipitated with ethanol, and the pellet was suspended in 10 mM Tris-Cl pH 8.0, 0.1 mM EDTA .  DNA probes with either a 5’-upstream or a 5’-downstream labeled end were generated by PCR using the appropriate combination of labeled and unlabeled primers with pUC119-SCP1 (Juven-Gershon et al., 2006) as template.  The ~220 basepair product was isolated by native gel electrophoresis, passively eluted into 10 mM Tris-Cl pH 8.0, 0.1 mM EDTA, 0.2% SDS, 1M LiCl  followed by phenol/CHCl3/IAA extraction, ethanol precipitation, and resuspension in 10 mM Tris-Cl pH 8.0, 0.1 mM EDTA .  A+G, T, and A chemical sequencing ladders used for mapping cleavage sites followed published methods (Iverson and Dervan, 1987; Sambrook, 1989).  
	Protein-DNA complexes were formed for 20 min. at 30°C in 20 μl of binding buffer containing 20 mM KHEPES pH 7.8, 4 mM MgCl2, 0.2 mM EDTA, 0.05% (v/v) NP-40, 8% (v/v) glycerol, 100 μg/mL BSA, and 1 mM (DNase I footprinting) or 2 mM DTT (MPE-Fe footprinting) upon final assembly containing 2 nM TFIID, 20 nM TFIIA, and 0.2 nM (DNase I footprinting) or 0.75 nM (MPE-Fe footprinting) DNA probe.  Proteins were diluted in a diluent consisting of binding buffer with 20% glycerol and 200 μg/mL BSA.  DNase I digestion was initiated by the addition of 2.2 μl of 0.38 – 0.75 mU/μl DNase I (in diluent containing 5 mM CaCl2) and terminated 30 seconds later by the addition of 158 μl of a stop solution (10 mM Tris-Cl pH 8.0, 3 mM EDTA, 0.2% SDS).  MPE-Fe(II) was generated by combining equal volumes of 1 mM methidiumpropyl EDTA (Sigma, discontinued) and 1 mM NH4Fe(II)SO4 (Aldrich) for 5 min. followed by dilution in water.  MPE-Fe(II) cleavage was initiated by the sequential addition of 1.2 μl 23 mM NaAscorbate and 1.2 μl 46 µM MPE-Fe(II) and terminated 2 min. later by the sequential addition of 2 μl 120 μM bathophenanthroline and 156 μl stop solution.  Samples were processed by phenol/CHCl3/IAA extraction and ethanol precipitation.  Digestion products were resolved on 10% polyacrylamide (37.5:1) containing 8.3 M urea until xylene cyanol migrated 80% down the gel.  The phosphor image (Typhoon Trio; GE Healthcare) was analyzed using Image Gage (Fuji Film). 
