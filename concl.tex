\chapter{Conclusions and future outlook}

The structural studies presented here for human TFIID reveal that it undergoes a dramatic reorganization of its lobe A. There are few other protein complexes that exhibit this type of dramatic structural rearrangement, where a 300 kDa sub-domain can be repositioned by 100\AA\ . By mapping the conformational landscape of lobe A within the context of activator (TFIIA, p53, c-Jun, and sp1) and DNA binding, this thesis has providing the structural basis for regulated promoter interactions between TFIID and DNA. Considering that TFIID has been studied for the past 30 years, this work is the first glimpse into the highly evolutionarily conserved activity of TFIID. \\     

\section{Discovery of the rearranged conformation}

\indent The flexibility of lobe A was previously overlooked because we assumed that TFIID would maintain a specific 3D shape. Since previous work in the lab had determined the three-dimensional structure of the canonical state \cite{Grob_1281,Liu_723,Liu_574}, we attempted to characterize the binding of TFIID to SCP DNA. After 3 - 4 years of unsuccessful attempts to determine the structure of TFIID-TFIIA-SCP within the lab, we finally had the breakthrough discovery that lobe A repositions itself within the DNA bound state. Like all important breakthroughs, this realization came through \emph{ab initio} 3D model calculations. \\
\indent Since all of our previous negative stain and cryo-EM models were unable describe the topology of the protein density seen within these DNA bound class averages, we attempted angular reconstitution of the cryo-EM 2D class averages. Working with Richard Hall, we calculated many different angular reconstitution models from the cryo-EM class averages of TFIID-TFIIA-SCP. These attempts were unsuccessful because we mistakenly tried to merge class averages from the canonical and rearranged state. Therefore, all of the resulting 3D models were unable to refine the cryo-EM data. \\
\indent The key control experiment was the implementation of OTR for TFIID-TFIIA-SCP. However, the first time we performed OTR, we merged class volumes before 3D model refinement. It is likely that the merged model had was created from the incorrect assumption that all of the 3D models came from the same conformation. Therefore, when this incorrectly merged model was refined against the untilted particle data, the resulting 3D model was no longer in agreement with the 2D reference-free averages, suggesting that something was wrong in our analysis. When I returned to this OTR dataset 6 months later, individual class volumes were \emph{not} merged together before 3D refinement.  Subsequent inspection of of the refined volumes revealed, much to our surprise, that lobe A adopted the rearranged state in almost all of the refined 3D structures. \\
\indent After discovering the rearranged state within the TFIID-TFIIA-SCP data, we were able to show that this new configuration was present in all TFIID datasets to date. The reproducibility and wide-spread presence of the rearranged state throughout all datasets of human TFIID confirmed that this reorganization was not an artifact of our analysis and likely important to TFIID function. It was reassuring to visualize the rearranged state from datasets collected before I was in the lab, in addition independently prepared samples of human TFIID by Weili Liu (Albert Einstein College of Medicine) \cite{Liu_723,Liu_574}. After measuring and calculating the conformational distribution of lobe A states for all of these datasets, we were able to measure how the presence of activators and DNA can shift the landscape in order to stabilize a particular conformation of TFIID.\\

\section{The rearranged state as \emph{the} DNA binding configuration of TFIID}

In addition to mapping the conformational landscape of TFIID, this work shows that the rearranged conformation likely serves as the general mode of promoter binding by TFIID. Footprinting experiments of TFIID on promoters containing mutations in the TATA box, Inr, and MTE/DPE showed that TFIID retains the same overall footprint on DNA regardless of the underlying sequence. For instance, we observed that the mutation of the MTE/DPE resulted in the loss of promoter binding by TFIID in the absence of TFIIA. However, this loss was fully restored when TFIIA was added to the reaction. This suggested that TFIIA anchors TFIID to the TATA box, allowing TFIID to bind the low affinity downstream DNA sites in a near-identical manner. Overall, these data indicate that the rearranged state serves as TFIID's conformation capable of multi-valent interactions with promoter DNA.\\
\indent EM structural studies have confirmed that TFIID binds promoters with varying architecture through the rearranged state. While we have not performed cryo-EM on all mutant promoters tested, cryo-EM of TFIID-TFIIA-SCP(mTATA) showed that it was bound through the rearranged conformation, similar in topology to TFIID-TFIIA bound to the wild-type SCP. Further support came from cryo-EM of an assembled TFIID-p53-TFIIA complex on native promoter DNA from the hdm2 gene. Even though this promoter only has a TATA box, preliminary cryo-EM visualization by Weili Liu (Albert Einstein College of Medicine) showed 2D reference-free class averages of TFIID bound to promoter DNA through the rearranged state. Considering the low affinity of TFIID alone for this hdm2 promoter, we propose that the role of activators (p53 and TFIIA) is to facilitate the formation of the rearranged state by anchoring TFIID to the promoter DNA. \\    
\indent Beyond the studies presented here, we believe that the characteristic footprinting pattern for human TFIID - observed for the past 30 years - corresponds to the rearranged state. Comparison of TFIID's footprinting patterns on the SCP(mMTE/DPE) and the previously published AdML promoter using partially purified (e.g. \cite{Sawadogo_3840,Va_3783}) or fully purified TFIID (e.g. \cite{Burke_3081,Chi_3023,Oelgeschlager_2880,Yakovchuk_306} shows the overall similarity of the results. In all cases, there is a TFIIA-mediated protection of the TATA that is accompanied by downstream contacts at positions +1 and +30 relative to the TSS. Therefore, we propose that the previously observed 'isomerizations' and 'structural changes' correspond to the formation of the rearranged state of TFIID bound to promoter DNA. \\
\indent Considering that the rearranged state appears to be the dominant DNA binding configuration of TFIID, the recognition of TFIID-TFIIA-SCP by TFIIB suggests that this state loads RNAPII at the TSS. While EM studies were able to unambiguously identify the binding site for TFIIB within the TFIID-TFIIA-TFIIB-SCP structure, biochemical footprinting showed that TFIIB makes intimate contacts with the flanking DNA sequences around the TATA box. Considering that the TATA box is located within lobe A, we believe that TFIIB is binding to lobe A. Furthermore, since the downstream contacts along the Inr and MTE/DPE were unchanged, TFIIB is binding to TFIID through the rearranged state. This indicates that, in addition to providing TFIID with high affinity DNA interactions, the rearranged state may serve as a structural signal to load RNAPII at the TSS. \\  

\section{Future directions}

\indent This thesis has laid the groundwork for mechanistic experiments to probe the contributions that TFIID makes during the formation of the pre-initiation complex. Additionally, for the first time, we have a detailed model of promoter binding and regulation of human TFIID activity that can be tested through a variety of experimental designs. The two most important research questions that remain will be obtaining a high resolution structure (10 - 15\AA\ ) of promoter-bound TFIID and determining the mechanism for RNAPII loading at the TSS.\\
 \indent High resolution studies of TFIID should be pursued through two orthogonal lines of research: 1) cryo-EM of a purified complex of TFIID-TFIIA-TFIIB-SCP and 2) biochemical isolation and EM analysis of the BC core. Given the ability of TFIIB to stabilize DNA contacts within TFIID-TFIIA-TFIIB-SCP, this complex should be co-immunoprecipitated with TFIID and pursued for high resolution studies. By visualizing this protein complex using high resolution optics (e.g. Titan EM microscopes by FEI), detectors (e.g. direct electron detectors), and phase-plate technologies, much higher levels of contrast should be possible for single particles of TFIID-TFIIA-TFIIB-SCP, allowing improved particle alignments. Especially considering the fact that we have these technologies installed on the shared Titan Microscope, a dedicated effort should be undertaken to get all of these devices working in concert to image single particles in vitreous ice.\\
\indent Considering the difficulties associated with using delicate pieces of equipment, an orthogonal strategy should be taken in order to increase the resolution of the BC core. Since the BC core appears to be the most well-ordered protein density within TFIID, biochemical experiments should be performed to try and isolate this subcomplex away from the holo-TFIID complex. A purified BC core will be a strong candidate for high resolution studies given its unique shape and apparent rigidity. To obtain such a sample, limited proteolysis should be performed with a wide-range of proteases to identify the proper conditions for specific cleavage of lobe A away from the BC core. It should be noted that previous work has shown that TAF1 (a proposed component of lobe A) appears to be sensitive to limited proteolysis, suggesting that it may be possible to separate lobe A from the BC core \cite{Ozer_1998}.\\
\indent Alternatively, recent work from Imre Berger's lab (EMBL, Grenoble) has purified and begun structural characterization of a subcomplex of human TFIID that comprises TAF6, -9, -4, -5, and -12. This work is currently unpublished, but after publication, the constructs should be requested and we should begin purification of these sub-complexes of TFIID. While the EM structure of this subcomplex does not appear (at first glance) to be related to the BC core, additional biochemical work and structural to define the relationship between the BC core and the reconstituted subcomplex should yield important information regarding the organization of subunits within TFIID.\\
\indent In addition to these high resolution structural studies of TFIID, mechanistic studies should be pursued with TFIID-TFIIA-TFIIB-SCP and RNAPII-TFIIF. Based upon the ability of Yuan He to purify and structurally characterize intermediates along the assembly pathway for the formation of a minimal pre-initiation complex nucleated by TBP, similar strategies should be taken to purified a complex of TFIID-TFIIA-TFIIB-SCP-RNAPII-TFIIF. While this is a non-trivial purification, it has the potential to offer unprecedented insight into the loading of RNAPII at the promoter. This will require extensive biochemical screening in addition to advanced image sorting methodologies, considering that there may be sample heterogeneity due to the presence of RNAPII.\\
