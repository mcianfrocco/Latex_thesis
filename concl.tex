\chapter{Conclusions and future outlook}

The structural studies presented here for human TFIID reveal that it undergoes a dramatic reorganization of its lobe A in a dynamic fashion. This flexibility was previously overlooked because we assumed that TFIID would maintain a specific 3D shape within the canonical state, an assumption that holds true for almost all proteins studied to date. After years of studying TFIID within the lab, we had the breakthrough discovery that lobe A moves dramatically between the canonical state and the newly identified rearranged state. This discovery was the result of careful re-analysis of OTR datafrom TFIID-TFIIA-SCP, where the refinement of individual class volumes showed that the majority of class volumes refined to the rearranged state. It should be noted that the first attempt of OTR using the Appion processing environment \cite{Lander_614}, where individual class volumes were merged together, resulted in refined 3D models that did not describe either the canonical or rearranged states (data not shown).  After the realization of lobe A's dramatic rearrangement, re-analysis of all negative staind and cryo-EM datasets confirmed that this rearrangement was \emph{always} occuring, but that we had simply ignored the rearranged 2D reference-free class averages.\\ 
\indent visualization in Weili's data; data within the alb
\indent general model of promoter binding (weili's data) on dna; USF papers from Roeder; ALSO TFIIB binding state
\indent future directions: high resolution studies of the BC core; phase plate; new detectors that allow dose fractionation;   
